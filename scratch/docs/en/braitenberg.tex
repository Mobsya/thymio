\chap{Braitenberg Creatures}\label{ch.brait}

\sect{What are Braitenberg creatures?}

\href{http://en.wikipedia.org/wiki/Valentino_Braitenberg}{Valentino
Braitenberg} was a neuroscientist who wrote a book describing
the design of virtual vehicles which exhibited surprisingly complex
behavior.\footnote{V. Braitenberg. \textit{Vehicles: Experiments in
Synthetic Psychology} (MIT Press, 1984).} Braitenberg's vehicles have
been widely used in educational robotics. Researchers at the MIT Media
Lab created hardware implementations of the vehicles called
\emph{Braitenberg creatures}.\footnote{David W. Hogg, Fred Martin,
Mitchel Resnick. \textit{Braitenberg Creatures}. MIT Media Laboratory,
E\&L Memo 13, 1991.
\href{http://cosmo.nyu.edu/hogg/lego/braitenberg_vehicles.pdf}{http://cosmo.nyu.edu/hogg/lego/braitenberg\_vehicles.pdf}.} The
vehicles were build from \emph{programmable bricks} that were the
forerunner of the LEGO Mindstorms robotics kits.

This document describes an implementation in Scratch of most of the
Braitenberg creatures from the MIT report. The implementation is based
upon a simulation in Scratch of the Thymio robot using the VPL
programming environment. It is intended to be used as an introduction to
Thymio / VPL for students with experience in Scratch.

The MIT hardware used light and touch sensors, while the Thymio robot
relies primarily on infrared proximity sensors. To enable comparison
with the MIT report, the names of the creatures used there have been
retained, even though they may not be appropriate for the Thymio
implementations. The order of presentation from the report has also been
retained, although this does not correspond to the difficulty of
implementation in either Scratch or VPL.

\newpage

\sect{Overview}

The project \p{template} contains the necessary sprites, the costumes
and the outline of the scripts. The Braitenberg projects can be built
starting with this project. 

The Thymio robot is simulated by a sprite called \p{Thymio} that appears
on the screen as:

\gr{thymio}{.2}

The sprite has seven costumes: the \p{blank} costume shown above; four
costumes (\p{center}, \p{right}, \p{left}, \p{rear}) with the
corresponding sensors lit, and two (\p{red}, \p{green}) with the top
lights turned on.

The \p{Thymio} sprite \emph{detects an object} when the mouse pointer is
close to the sprite. A new block called \p{get-pointer-direction}
returns in the variable \p{direction-to-pointer} the angle from the
sprite to the mouse pointer.\footnote{The implementation is described in
Appendix~\ref{ch.implementation}, except that the direction is to the
mouse pointer and not to a pointer sprite.} Zero degrees is the
direction that the \p{Thymio} faces and the angles increase clockwise.
The following test script in the template project demonstrates how the
simulation works:

\gr{template}{.5}

Start the program by clicking on the green flag. If the mouse pointer is
near the \p{Thymio} sprite, the sprite says the direction to the
pointer.

\newpage

\sect{Specification of the creatures}

\begin{description}

\item[Timid] When the robot does not detect an object, it moves forwards.
When it detects an object, it stops.

\item[Indecisive] When the robot does not detect an object, it moves
forwards. When it detects an object, it moves backwards. Experiment with
the definition of the distances for ``detect'' and ``not detect'' so
that the robot \emph{oscillates}: move forwards and backwards in
succession.

\item[Paranoid] When the robot detects an object, it moves forwards. When
it does not detect an object, it turns to the left.

\textbf{Paranoid1} When an object is detected by the center sensor of
the robot, it moves forwards. When an object is detected by the right
sensor, it turns right. When an object is detected by the left sensor,
it turns left.

\item[Dogged] When the robot detects an object in front, it moves
backwards. When the robot detects an object in back, it moves forwards.
When an object is not detected, it stops.

\item[Insecure] If an object is detected by the left sensor, the robot
turns right. If an object is not detected by the left sensor, it turns
left. Experiment with the angles of the turns until the robot can track
the mouse pointer placed ahead and to its left.

\item[Driven] If an object is detected by the left sensor, the robot
turns left. If an object is detected by the right sensor, it turns
right. The robot will approach the mouse pointer in a zigzag.

\item[Persistent] The robot moves forwards until it detects an object.
It then moves backwards for one second and reverses to move forwards
again.

\item[Attractive and repulsive] When an object approaches the robot from
behind, the robot moves forward until the object is no longer detected.

\item[Consistent] The robot cycles through four states when it is
clicked on: moving forwards, turning left, turning right, moving
backwards.

\item[Frantic] The top light flashes red.

\item[Observant] The robot turns the top light green when the right
sensor detects an object. The robot turns the top light red when the
left sensor detects an object. Once a light is turned on, the robot
waits three seconds before turning off; during this period, the light
does not change.

\end{description}
