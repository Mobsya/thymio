\chap{Defining New Blocks}\label{ch.new}

\scrprj{likes}

The project \bu{likes} is similar to \bu{obeys} except that the
\p{Thymio} sprite turns towards the object when one is detected by the
left or right sensors and not just by the center sensor:

\begin{footnotesize}
\begin{verbatim}
if center sensor faces Pointer
    switch costume to center
    go to Pointer
else
  if right sensor faces Pointer
      switch costume to right
      go to Pointer
  else
    if left sensor faces Pointer
      switch costume to left
      go to Pointer
\end{verbatim}
\end{footnotesize}

In Scratch, we can define new blocks such as
\scrblk[-8]{go-to-pointer-block}. Once the block is defined, we can use it
to write the script for \bu{likes} (\cref{fig.likes}).

\begin{figure}
\gr{likes}{.75}
\caption{Script for \bu{likes}}\label{fig.likes}
\end{figure}

How do we define the new block? Go to the palette \bu{More Blocks} and
click on \scrblk{make-block}. Enter the name of the new
block, \p{go-to-pointer}, in the purple field in the \p{New block} window
that is opened. This will create the block \scrblk[-10]{define-go-to-pointer}
block in the script area and the \scrblk[-8]{go-to-pointer-block} in the block
area for this palette. Now you can drag-and-drop the blocks required to
implement the new block (\cref{fig.go-to}).

\trickbox{When defining a new block, click on \bu{Options} and \bu{Run
without screen refresh}. This ensures that the new block is run as one
action and the user does not see the result of running each block in the
definition.}

The new block \scrblk[-8]{go-to-pointer-block} can be used in different
scripts and even more than once in the same script as shown in
\cref{fig.likes}. Clearly, this script is much shorter than it would be
if we had to copy the blocks again and again.

\begin{figure}
\gr{go-to-pointer}{.5}
\caption{Definition of the block \bu{go-to-pointer}}\label{fig.go-to}
\end{figure}
