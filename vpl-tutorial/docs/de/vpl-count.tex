\chap{Zählen (Fortgeschrittener Modus)}\label{ch.counting}

In diesem Kapitel zeigen wir, wie die Zustände des Thymio Roboters benutzt werden können, um zu zählen und einfache Arithmetik zu betreiben.

Das Design und die Einführung des Projektes wird nicht im Detail präsentiert. Wir nehmen an, dass Sie inzwischen genügend Übung haben, um diese selbständig zu entwickeln. Der Quellcode eines funktionierenden Programms ist zwar im Archiv abgelegt, dies sollten Sie aber erst konsultieren, falls Sie wirklich auf ein Problem stossen, das Sie nicht lösen können.
Dieses Projekt benutzt das Ereignis \blksm{event-clap}, um den Zustand zu ändern und 
das voreingestellte Verhalten der LED Lichter, um die Zustände im Kreis anzuzeigen. Natürlich können Sie beide Verhaltensweisen anpassen.

\importantbox{Der Ist-Zustand des Roboter wird im LED Kreis auf der Oberseite des Roboters angezeigt.	\cref{fig.state-leds} zeigt den Roboter im Zustand \bu{(ein,ein,ein,ein)}.}


\sect{Gerade und ungerade Zahlen}

\begin{quote}
	\textbf{Programmieren}\\Wählen Sie ein Viertel der Zustandsanzeige. Es wird \bu{aus} (weiss) anzeigen, wenn Sie eine gerade Anzahl Mal in die Hände klatschen und \bu{ein} (orange) anzeigen, wenn Sie eine ungerade Anzahl Mal in die Hände klatschen. 
	Wenn Sie den mittleren Knopf drücken, wird der Zustand auf gerade zurückgestellt (da 0 ja ebenfalls eine gerade Zahl ist).
\end{quote}

{\raggedleft \hfill Beispielprogramm \bu{count-to-two.aesl}}

Unsere Methode zu zählen (gerade und ungerade), zeigt das Konzept der \emph{Modulo-2-Arithmetik} auf. Dabei zählen wir von 0 (gerade) auf 1 (ungerade) und dann zurück zu 0. Es handelt sich um die Ganzzahldivision (teilen mit Rest); die Operation modulo wird in der Regel mit ''mod'' abgekürzt. Der Ausdruck \emph{Modulo} entspricht dem Ausdruck \emph{Rest}: Falls 7 Mal in die Hände geklatscht wurde, wird die 7 durch 2 geteilt, dies ergibt 3 Rest 1 (7 mod 2 = 3 Rest 1). Nun merken wir uns nur den Rest 1. 

Ein anderer Ausdruck für das selbe Konzept ist die \emph{zyklische Arithmetik}. Statt von 0 auf 1 und dann von 1 auf 2 zu zählen, wird im \emph{Kreis} wieder zurück zum Anfang gezählt: 0, 1, 0, 1, \ldots.

Diese Konzepte sind uns aus dem Alltag sehr vertraut, da sie auch bei der Zeitmessung verwendet werden. Minuten und Sekunden werden mod 60 gezählt und Stunden mod 12 oder mod 24. Deshalb bezeichnen wir die Sekunde nach der 59. Sekunde nicht mit 60 sondern mit 0. Wir zählen im Kreis und beginnen wieder bei 0. Ähnlich die Stunden: Nach 23 kommt nicht 24, sondern 0. Wenn es 23:00 Uhr ist und wir machen ab, uns in 3 Stunden zu treffen, dann ist die abgemachte Zeit für das Treffen 23+3 = 26 mod 24, was 02:00 Uhr in der Nacht ist.

\sect{Unäres Zählen}

\begin{quote}
Ändern Sie das Programm, um mit mod 4 zu zählen. 
Es gibt vier mögliche Reste: 0, 1, 2, 3. 
Wählen Sie drei der Zustands-Viertel, wovon jeder einen Rest-Wert repräsentiert, also 1, 2 oder 3; der Wert 0 soll durch den Zustand ''alle Viertel'' auf \bu{aus} repräsentiert werden.
\end{quote}

Diese Methode der Repräsentation von Zahlen wird \emph{unäre Repräsentation} genannt: die Zahlen werden durch verschiedenen Elemente dargestellt, denen verschiedene Zustände zugeordnet sind. Wir benützen die unäre Repräsentation zum Beispiel im Alltag, wenn wir etwas während eines Vorgangs immer weiter zählen (z.B. wie viele Drehungen hat der Roboter gemacht). Zum Beispiel:

\begin{picture}(35,10)
\multiput(5,0)(5,0){4}{\put(0,0){\line(0,1){10}}}
\put(0,0){\line(3,1){25}}
\put(32,0){\line(0,1){10}}
\end{picture}
bedeutet 6.

{\raggedleft \hfill Beispielprogramm \bu{count-to-four.aesl}}

\exercisebox{\thechapter.1}{Wie hoch können wir mit Thymio zählen, wenn wir die unäre Repräsentation verwenden?}


\sect{Binäres Zählen}

Wir sind vertraut mit verschiedenen \emph{Zahlensystemen} (oder Stellenwertsystemen), speziell mit dem Dezimalsystem (Basis 10). Die Symbole 256 im Dezimalsystem repräsentieren nicht drei verschiedene Objekte. Stattdessen repräsentiert die 6 die Einer, die 5 die Zehner und die 2 die Hunderter. Erst durch das Hinzufügen dieser Faktoren erhalten wir die Zahl ''Zweihundertsechsunfüngzig''  (2$\times$100 + 5$\times$10 + 6$\times$1 = 256). Indem wir die Basis 10 verwenden, können wir sehr grosse Zahlen in einer kompakten Darstellung darstellen. Zudem ist die Arithmetik mit grossen Zahlen relativ einfach; man benutzt die Methoden aus der Schule.

Wir benutzen die 10er Repräsentation, weil wir 10 Finger haben. So ist es einfach diese Repräsentation zu lernen. Computer dagegen haben nur zwei ``Finger'' (\bu{aus} und \bu{ein}).
Deshalb wird die binäre oder ''Basis-2''-Arithmetik beim Programmieren verwendet. Diese Arithmetik ist uns anfangs fremd, weil ähnliche Symbole 0 und 1 wie im Dezimalsystem zur Basis 10 benutzt werden. Die Regel beim zählen wird zyklisch bei der 2 und nicht erst bei der 10:

\begin{displaymath}
0, 1, 10, 11, 100, 101, 110, 111, 1000, \ldots
\end{displaymath}

Wenn man die Binärzahl 1101 betrachtet, lesen wir sie von rechts nach links. Die erste Zahl ganz rechts repräsentiert die Zahl der Einer (1), die nächste Zahl Zweier (2 = 1$\times$2), danach die Vierer, da 1$\times$2$\times$2 = 4 und die Zahl ganz links die 1$\times$2$\times$2$\times$2=8 also Achter. 
1101 repräsentiert deshalb je einen Einer, Vierer und Achter aber keinen Zweier oder die Zahl 1+0+4+8, was im Dezimalsystem Dreizehn.

\begin{quote}
\textbf{Programmieren}\\
Verändern Sie das Programm mit dem Sie mod 4 gezählt haben in eine binäre Repräsentation.
\end{quote}

{\raggedleft \hfill Beispielprogramm \bu{count-to-four-binary.aesl}}

Wir benötigen nur zwei der Zustands-Viertel, um die Zahlen 0--3 mit binären Zahlen auszudrücken. Legen wir fest, dass das Viertel rechts oben die Einer repräsentiert, \bu{aus} (weiss) für Null und \bu{ein} (orange) für Eins und das Viertel links oben repräsentiert die Zweier. So repräsentiert zum Beispiel der Zustand \blksm{state-right} die Zahl 1 und der Zustand \blksm{state-left} die Zahl 2.
Falls beide Viertel weiss sind, repräsentiert der Zustand die Zahl 0 und wenn beide Viertel orange sind, repräsentieren die Zustände die Zahl 3.

Es gibt vier Übergänge (Zustandsdiagramm): $0\rightarrow 1, 1\rightarrow 2, 2
\rightarrow 3, 3\rightarrow 0$, so werden vier Ereignis-Aktions-Paare benötigt und zusätzlich noch ein Paar, um das Programm in den Anfangszustand zurückzuversetzen, wenn der mittlere Knopf gedrückt wird.

\bigskip

\informationbox{Unbenutzte Vierter ignorieren}{Die zwei unteren Viertel werden nicht benutzt, deshalb bleiben sie grau und werden durch das Programm ignoriert.}

\bigskip

\exercisebox{\thechapter.2}{Erweitere das Programm, dass es in mod 8 zählt. Das untere linke Viertel repräsentiert die Anzahl der 4er.}

\exercisebox{\thechapter.3}{Wie hoch können wir mit dem Thymio zählen, wenn wir binäre Repräsentation verwenden?}

\sect{Addition und Subtraktion}

Ein Programm zu schreiben, um bis 8 zählen zu können ist ziemlich mühsam, da wir 8 Ereignis-Aktions-Paare programmieren müssen, eines für jede Umschaltung von $n$ nach $n+1$ (modulo 8). 
Natürlich zählen wir nicht so in einem Basiszahlensystem, stattdessen haben wir Methoden um Additionen auszuführen, indem Ziffern an gleicher Stelle addiert werden und allenfalls ein Übertrag eingerichtet wird. 
Im Zahlensystem mit der Basis 10 wird dies so dargestellt:

\begin{displaymath}
\begin{array}{r}
387\\
+426\\
\rule[1pt]{1.5em}{1pt}\\
813\\
\end{array}
\end{displaymath}
und auf die gleiche Weise im Zahlensystem mit Basis 2:
\begin{displaymath}
\begin{array}{r}
0011\\
+1011\\
\rule[1pt]{2em}{1pt}\\
1110\\
\end{array}
\end{displaymath}

Wenn wir 1 und 1 addieren, ergibt das nicht 2 sondern 10. Die 0 wird in der selben Spalte geschrieben und die 1 wird in die links davon liegende Spalte übertragen. Das Beispiel oben zeigt die Addition von 3 =(0011) und 11(=1011) was 14 ergibt (=1110).

\begin{quote}
\textbf{Programmieren}\\
Schreiben Sie ein Programm, das mit der Repräsentation von 0 beginnt. Jedes Mal, wenn man in die Hände klatscht, zählt das Programm 1 zur Zahl hinzu. Die Addition sei mod 16, deshalb ergibt 15 addiert um 1 gleich 16 mod 16 = 0.
\end{quote}

{\raggedleft \hfill Beispielprogramm \bu{addition.aesl}}

\textbf{Anleitung:}

\begin{itemize}
	\item Beginnen Sie im Viertel oben rechts und gehen Sie im Gegenuhrzeigersinn weiter. Die Viertel repräsentieren die Anzahl der Einer, Zweier, Vierer und Achter in der gegebenen Zahl. Das Viertel unten rechts wird also gebraucht, um die 8er zu repräsentieren.
	\item Wenn das rechte obere Viertel die 1er repräsentiert und 0 (weiss) anzeigt, ändere dies einfach auf 1 (orange). Mach dies egal was die anderen Viertel anzeigen.
	\item Wenn das obere rechte Viertel die 1er repräsentiert und 1 (orange) anzeigt, ändere dies auf 0 (weiss) und behalte dann 1. Das gibt drei Ereignis-Aktions Paare, abhängig von der Anordnung des  \emph{nächsten} Viertels, das 0 (weiss) anzeigt. 
	\item Wenn alle Viertel 1 anzeigen (orange) wird 15 repräsentiert, wird 1 zu 15 modulo 16 addiert, ergibt dies 0, repräsentiert indem alle Viertel 0 (weiss) anzeigen.
\end{itemize}

\bigskip

\exercisebox{\thechapter.4}{Ändern Sie das Programm so, dass es mit 15 beginnt und mit jedem Mal Klatschen 1 subtrahiert bis 0 und dann wieder bei 15 beginnt.}

\bigskip

\exercisebox{\thechapter.5}{Platzieren Sie eine Sequenz mehrerer kurzer schwarzer Isolierband-Streifen auf eine helle Oberfläche (oder helle Streifen auf einer dunklen Oberfläche). Schreiben Sie ein Programm, welches den Roboter vorwärts bewegen und stoppt lässt, sobald er das vierte Klebeband erreicht hat.}

\bigskip

Diese Aufgabe ist nicht einfach: Die Klebebandstreifen müssen genügend weit auseinander platziert werden, damit der Roboter sie entdeckt, aber nicht so weit, dass nicht mehr als ein Ereignis pro Streifen stattfindet. Sie müssen allenfalls mit der Geschwindigkeit des Roboters experimentieren.
