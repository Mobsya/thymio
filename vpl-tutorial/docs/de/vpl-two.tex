\chap{Mehrere Thymios}\label{ch.two}

Sie können zwei oder mehrere Thymio Roboter gleichzeitig steuern.

\textbf{Spezifikation}

Setzen Sie zwei Roboter T1 und T2 so, dass sie sich gegenüber stehen; T1 ist der Jäger, T2 ist die Beute. Wenn T1 erkennt, dass er nahe bei T2 ist, hält er an; wenn T2 bemerkt, dass er nahe bei T1 ist, zieht er sich zurück bis er T1 nicht mehr erkennt.  

\textbf{Anleitung}

\begin{itemize}

\item Die Programme für T1 und T2 haben zwei Ereignis-Aktions-Paare: eines, dessen Ereignis das Erkennen eines Objekts mit dem mittleren horizontalen Sensor ist und ein anderes, des Ereignis das Nicht-Erkennen eines Objekts ist. Die Aktionen für T1 und T2 sind allerdings unterschiedlich. 

\item Verbinden Sie zwei Thymios T1 und T2 mit Ihrem Computer und schalten Sie sie ein. Starten Sie zwei Kopien von VPL. Im Ziel-Auswahl-Fenster (siehe Bild ~\ref{fig.connect}) sollten beide Roboter T1 und T2 erscheinen; wählen Sie T1 in der einen Kopie von VPL bzw. dem Ziel-Auswahl-Fenster und T2 im anderen. Öffnen und starten Sie in der ersten Kopie von VPL das Programm \p{chase} (Jäger) und im anderen das Programm \p{retreat} (Beute).
 
\end{itemize}

\textbf{Experimente}

\begin{itemize}

\item Was passiert, wenn Sie die Programme austauschen: auf T1 läuft 
\p{retreat} und auf T2 läuft \p{chase}? Erklären Sie!

\item Experimentieren Sie im fortgeschrittenen Modus mit unterschiedlichen Werten für die Schwellwerte. 

\end{itemize}

%\bigskip

{\raggedleft \hfill Beispielprogramme \bu{chase.aesl}, \bu{retreat.aesl}}

\bigskip

\informationbox{Kommunikation zwischen Robotern}{Thymio Roboter können sich Meldungen schicken. Diese Funktionalität wird in der AESL-Sprache und -Umgebung unterstützt.}
