\chap{Los, beweg dich}\label{ch.moving}

\sect{Vorwärts und rückwärts fahren}

Thymio hat zwei Motoren mit denen er seine zwei Räder unabhängig voneinander antreiben kann. Beide Motoren können vorwärts und rückwärts drehen. Dadurch kann der Roboter vorwärts und rückwärts fahren. Wir wollen uns mit einem kleinen Projekt befassen, um mehr über die Motoren zu lernen.

Der Aktions-Block für die Motoren \blksm{action-motors} zeigt ein kleines Bild des Roboters in der Mitte mit zwei Schiebereglern links und rechts. Mit den beiden Balken können Sie die Geschwindigkeit der beiden Motoren einstellen, mit dem linken Balken die des linken Motors und mit dem rechten Balken die des rechten Motors. Wenn das weisse Quadrat in der Mitte ist (vertikal), dreht der Motor nicht.  Sie können die Geschwindigkeit ändern, indem Sie das weisse Quadrat verschieben. Schiebt man das Quadrat nach oben, dreht der Motor vorwärts --- je weiter oben, desto schneller; schiebt man es nach unten, dreht der Motor entsprechend rückwärts.

Erstellen Sie ein Programm, um den Roboter vorwärts fahren zu lassen, wenn der Vorwärts-Kopf gedrückt wird und rückwärts, wenn der Rückwärts-Knopf gedrückt wird.

{\raggedleft \hfill Beispielprogramm \bu{moving.aesl}}

Wir brauchen zwei Ereignis-Aktions-Paare:
%(\cref{fig.nostop}).

%\begin{figure}
\begin{center}
\gr{no-stop-motors}{.3}
%\caption{Moving forward or backwards}\label{fig.nostop}
\end{center}
%\end{figure}

Ziehen Sie die Ereignis- und Aktions-Blöcke in die Programmierumgebung und stellen Sie für beiden Motoren die Schieberegler auf halbe Geschwindigkeit ein. Dies indem Sie die Quadrate für vorwärts fahren halb nach oben und für rückwärts fahren halb nach unten verschieben.

Führen Sie das Programm aus und drücken Sie die Knöpfe, um den Roboter vorwärts- und rückwärts fahren zu lassen. 

\sect{Den Roboter anhalten}

\textbf{Hilfe!} Ich kann den Roboter nicht mehr stoppen!

Klicken Sie auf das Symbol \blksm{stop}, um den Roboter zu stoppen.

Wir wollen diese Problem beheben, indem wir ein neues Ereignis-Aktions-Paar
hinzufügen: \blkc{stop-motors}
Dieses soll die Motoren stoppen, wenn der mittlere Knopf gedrückt wird. Wenn Sie den Motoraktionsblock in die Programmierumgebung ziehen, sind die Schieberegler in der Mitte, was die Motoren ausschaltet und den Roboter anhalten lässt.

\sect{Nicht vom Tisch fallen!}

Wenn der Roboter auf dem Boden fährt, kann er im schlimmsten Fall in eine Wand
fahren oder sein USB-Kabel herausziehen. Aber wenn der Roboter auf einem Tisch
fährt, kann er auf den Boden fallen und kaputt gehen! Wir wollen den Roboter so
programmieren, dass er anhält, sobald er die Tischkante erreicht.

\warningbox{Wenn der Roboter auf einem Tisch fährt, muss man bereit sein, um ihn aufzufangen, falls er hinunterfällt.}

Drehen Sie Ihren Thymio auf den Rücken. Nun sehen Sie, dass er unten zwei
kleine, schwarze Rechtecke mit optischen Elementen hat (\cref{fig.bottom}).
Das sind \emph{Bodensensoren}. Diese senden Infrarotlichtimpulse aus und messen wie viel Licht reflektiert wird. 
Auf einem hellen Tisch wird viel Licht reflektiert. Fährt der Roboter über die Tischkante wird wenig Licht reflektiert. Tritt dies ein, möchten wir, dass der Roboter stoppt.

\trickbox{Benutzen Sie einen hellen Tisch aber keinen aus Glas, da von diesem kein Licht reflektiert wird. Thymio kann dann nicht erkennen, ob er auf einem Tisch ist oder nicht.}

Ziehen Sie das Bodensensor-Ereignis \blksm{event-prox-ground}
in Ihr Programm. Oben hat dieser Block zwei kleine Quadrate. Wenn Sie sie anklicken, werden sie weiss mit rotem Rand, schwarz und dann wieder grau. Die Farben haben verschiedene Bedeutungen:

\begin{itemize}

\item \textbf{Grau}: Der Sensor wird nicht beachtet.

\item \textbf{Weiss}: Das Ereignis tritt ein, wenn viel Licht reflektiert wird. \label{p.proximity-colors1} Links vom weissen Feld wird ein kleiner roter Punkt dargestellt; dies nimmt Bezug auf das rote Licht neben dem Sensor, das leuchtet, wenn der Sensor etwas entdeckt hat.\footnote{Das weisse Quadrat hat einen roten Rand, der daran erinnert, dass das Ereignis eintrifft, wenn die Lichter neben dem Sensor aufleuchten.}

\item \textbf{Schwarz}: Das Ereignis tritt ein, wenn es wenig oder kein reflektierendes Licht gibt.

\end{itemize}

%\trickbox[Information]{The gray, red and white colors used in a block
%are arbitrary and others could have been chosen.}

Damit der Roboter an der Tischkante anhält (wo es nur geringe Reflektion von Licht gibt) klicken Sie beide Quadrate bis sie schwarz werden.
Erstellen Sie das folgende Ereignis-Aktions-Paar: \blkc{dont-fall}

\begin{figure}
\begin{center}
\gr{bottom}{0.6}
\caption{Thymios Unterseite mit den beiden Bodensensoren}\label{fig.bottom}
\end{center}
\end{figure}

Stellen Sie den Roboter in der Nähe der Tischkante auf (auf die Tischkante ausgerichtet) und betätigen Sie den vorderen Knopf. Der Roboter sollte sich auf die Kante zu bewegen und anhalten, bevor er vom Tisch fällt. 

\bigskip

\exercisebox{\thechapter.1}{ Experimentieren Sie mit der Geschwindigkeit. Gelingt es dem Roboter bei Höchstgeschwindigkeit doch noch rechtzeitig anzuhalten? Falls nicht: ab welcher Geschwindigkeit fällt er vom Tisch? Kann man den Roboter auch im Rückwärtsgang davon abhalten, vom Tisch zu fallen?}

\bigskip

\warningbox{Als wir dieses Programm getestet haben, \emph{ist} der Roboter vom Tisch gefallen. Der Grund war, dass der Tisch eine abgerundete Kante hatte; sobald die fehlende Reflektion erkannt wurde, war es schon zu spät und der Roboter kippte nach vorne, verlor an Stabilität und viel vom Tisch.}
