\chap{The Rabbit and the Fox}\label{ch.rabbit}

This chapter contains the specification of a large project (my program
uses 7 event-actions pairs, each with 2--3 actions). You should have
enough experience by now designing and implementing VPL programs in
order to write it yourself. We will give the specification of the
behavior of the robot as a list of tasks and suggest that you develop
the program by implementing each task in turn.


\textbf{Story}\footnote{The story is loosely inspired by a
\href{http://www.cs.hmc.edu/~fleck/parable.html}{joke}
well-known to PhD students.} The robot is a rabbit,
walking in the forest. A fox chases the the rabbit to catch
it from behind. The rabbit senses the fox, turns around and catches the
fox.


\textbf{Specification}

For each event, we specify a top color to be displayed when the event occurs.

\begin{enumerate}
\item Touch the forwards button: the robot moves forwards (blue).
\item Touch the backwards button: the robot stops (off).
\item If the robot detects the edge of the table it stops (off).
\item If the left rear sensor detects an object, the robot quickly turns
left (counterclockwise) until the object is detected by the front center
sensor (red).
\item If the right rear sensor detects an object, the robot quickly turns
right (clockwise) until the object is detected by the front center
sensor (green).
\item When the object is detected by the front center sensor, the robot
moves forward quickly for one second (yellow) and then stops (off).
\end{enumerate}

{\raggedleft \hfill Program file \bu{rabbit-fox.aesl}}
