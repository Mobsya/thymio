% !TeX root = vpl.tex

\chapter*{VPL Tutorial Version History}

\paragraph{Version 1.5}

\begin{itemize}

\item Events are now implemented using the AESL programming language structure \p{when} instead of \p{if}. This means that events will occur only when the event initially occurs and not repeatedly, as explained on page~\pageref{p.if-when}. This change might cause unexpected behavior in some programs described in this tutorial.

\item Dynamic feedback of the execution is implemented (page~\pageref{p.feedback}).

\end{itemize}

\paragraph{Version 1.4}

\begin{itemize}

\item The graphic design of the buttons for the blocks has been changed,
primarily to support additional features.

\item In 1.3, a \emph{red} box in the event block for a horizontal
sensor caused an event when the sensor detected an object, while a
\emph{white} box caused an event when there was no object in front of
the sensor. In 1.4, a \emph{white} box causes an event when a lot of
reflected light is detected from an object, while a \emph{black} box
causes an event when little or no reflected light is detected because
there is no object in front of the sensor
(pages~\pageref{p.proximity-colors1},~\pageref{p.proximity-colors2}).
The ground sensors also use white and black, instead of white and red,
but the behavior in 1.4 is the same as in 1.3 except that black is used
instead of red.

\item In advanced mode, the thresholds of sensors can be
set (page~\pageref{p.proximity-sensitivity}).

\item In advanced mode, an event can be associated with ranges of values
of the forward/backward and left/right accelerometers
(page~\pageref{p.accel}).

\item Multiple actions associated with an event (page~\pageref{p.multiple}).

\item Blocks and event-actions pairs can be copied
(page~\pageref{p.copy-pairs}).

\item Screenshots of VPL programs can be exported in several
 graphics formats (page~\pageref{p.export}).

\item Undo/Redo buttons have been added (page~\pageref{p.undo}).

\item The Run button blinks green when the program has been changed
(page~\pageref{p.blink}).

\item It is no longer possible to change the color scheme of VPL.

\end{itemize}
