\chap{Quelques trucs pour programmer avec VPL}\label{a.tips}

\sect{Explorer et expérimenter}

\begin{description}

\item[Comprendre chaque bloc événement et action]
Pour chaque bloc événement et chaque bloc action,
prenez-vous le temps de faire plusieurs essais jusqu'à ce que vous compreniez
exactement comment il fonctionne.
Pour mieux comprendre comment un bloc action fonctionne,
créez une paire avec un bloc événement bouton.
C'est un événement facile qui vous permettra de bien discerner l'action du bloc action.
Pour les blocs événement, vous pouvez construire une paire avec l'action changer de couleur.

\item[Tester les blocs événement capteurs]
Les petites lumières rouges à côté de chaque capteur permettent de voir quand ce capteur détecte
un objet.
Déplacez par exemple vos doigts devant les capteurs et regardez quelles lumières sont allumées,
ce qui indique quels capteurs détectent vos doigts.
Créez une paire événement-actions qui consiste en un événement capteur et l'action couleur du haut
et jouez avec les différents réglages des petits carrés dans le bloc événement (gris, blanc, noir et
gris foncé en mode avancé).

\end{description}

\sect{Construisez un programme}

\begin{description}

\item[Faites un plan pour votre programme]
Avant d'écrire un programme,
commencez par rédiger une description du fonctionnement voulu de votre programme:
une phrase pour chaque paire événement-actions.

\item[Construisez une paire événement-actions à la fois]
Une fois que vous aurez compris comment chacune des paires événement-actions fonctionnent,
vous pourrez les mettre ensemble pour créer votre programme.

\item[Testez chaque nouveauté de votre programme]
Testez votre programme à chaque fois que vous ajoutez une nouvelle paire événement-actions
pour que vous puissiez rapidement trouver la paire à l'origine d'une erreur dans votre programme.

\item[Utilisez \bu{Sauvegarder sous} quand vous avez changé votre programme]
Avant de modifier votre programme, cliquez sur \blksm{saveas} pour sauvegarder votre programme
en utilisant un autre nom. Si votre modification devait ne pas marcher,
il vous sera alors facile de revenir à la version précédente.

\item[Montrez ce qui se passe]
Utilisez des couleurs pour montrer ce que le programme est en train de faire.
Par exemple, si un capteur d'une paire a une action associée tourner à gauche
et un capteur d'une autre paire a une action associée tourner à droite,
ajoutez une action aux deux paires qui affichent des couleurs différentes.
Vous pourrez ainsi voir si le problème est dû aux capteurs ou si ce sont les moteurs
qui ne réagissent pas correctement aux événements capteurs.

\end{description}


\sect{Régler les problèmes}

\begin{description}

\item[Utilisez une surface lisse]
Vérifiez que votre Thymio se déplace sur une surface lisse et propre.
Les moteurs risquent sinon de ne pas réussir à déplacer le robot,
ou alors les virages risquent de ne pas être réguliers.

\item[Utilisez un long cable]
Assurez-vous d'avoir un cable assez long.
Si le robot va loin, le cable pourrait sinon freiner ou arrêter le robot.

\item[Les événements capteurs peuvent ne pas être déclanchés]
Les événements capteurs sont déclanchés 10 fois par seconde.
Si le robot se déplace très rapidement, il est possible qu'un événement ne soit pas déclanché.

Par exemple, si le robot est censé s'arrêter lorsqu'il détecte le bord de la table
mais qu'il avance très rapidement, il est possible qu'il tombe de la table
avant que l'événement capteurs n'arrête les moteurs.
Lorsque vous lancer un programme, commencez avec une vitesse lente.
Vous pouvez ensuite l'augmenter graduellement.

Comme autre exemple, considérez le programme du \cref{ch.line} où Thymio suivait une ligne.
L'algorithme de ce programme repose sur la capacité à détecter le moment où
un des capteurs voit la ligne et l'autre ne la voit plus.
Si le robot avance trop rapidement, le moment où seul un capteur détecte la ligne 
est trop bref pour déclancher un événement.

\item[Problèmes avec les capteurs du bas]
Dans les programmes comme celui à peine mentionné,
le capteur doit pouvoir distinguer beaucoup de lumière réfléchie de peu de lumière réfléchie.
Veillez à avoir un grand contraste entre les deux.
Si votre table, par exemple, n'est pas assez claire, fixer des feuilles blanches 
permettra d'obtenir de meilleurs résultats.

Vous pouvez sinon aussi ajuster les seuils de détection en mode avancé.

\item[Les paires événement-actions sont exécutées à la suite]
En théorie, les paires événement-actions sont exécutées \emph{simultanément}---en même temps;
en pratique, elles sont exécutées à la suite, dans l'ordre dans lequel elles apparaissent 
dans votre programme.
Comme on peut le voir dans l'exercice~4.2, ceci peut poser problème car la seconde action
peut entrer en conflit avec ce qui a été exécuté par la première action.

\item[Problèmes avec l'événement frapper des mains]
\emph{N'utilisez pas} l'événement frapper des mains \blksm{event-clap} lorsque les moteurs tournent.
Les moteurs génèrent beaucoup de bruit et peuvent provoquer des événement frapper des mains
indésirables.

De même, \emph{n'utilisez pas} l'événement tappe \blksm{event-tap} et l'événement frapper des mains dans le même programme.
En effet, donner une tappe au robot génère du bruit que le robot peut interpréter comme une frappe
des mains.

\end{description}
