\chapter*{Versions 1.3 et 1.4}

La liste suivante est destinée aux lecteurs ayant déjà une certaine expérience avec VPL dans sa version précédente, Aseba 1.3; elle énumère les différences entre cette version et la 1.4.
Si vous n'avez jamais utilisé VPL auparavant, vous pouvez sauter cette section.

\textbf{Changements apportés aux blocs événement et action}

\begin{itemize}

\item L'apparence graphique des boutons des blocs a été modifiée, principalement dans le but d'intégrer
de nouvelles fonctionnalités.

\item Dans la 1.3, une case \emph{rouge} dans un bloc événement capteurs horizontaux déclanchait un événement
lorsque le capteur détectait un objet, alors qu'une case \emph{blanche} déclanchait un événement lorsqu'il n'y
avait pas d'objet en face du capteur.
Dans la 1.4, une case \emph{blanche} déclanche un événement quand le capteur mesure beaucoup de lumière réfléchie,
alors qu'une case \emph{noire} déclanche un événement lorsque le capteur mesure peu ou pas de lumière réfléchie
parce qu'il n'y a pas d'objet devant le capteur (pages~\pageref{p.proximity-colors1},~\pageref{p.proximity-colors2}).
Les capteurs du bas utilisent aussi les couleurs noir et blanc, au lieu de rouge et blanc,
mais le fonctionnement dans la 1.4 est identique à la 1.3, avec le noir au lieu du rouge.

\item En mode avancé, les seuils des capteurs peuvent être modifiés (page~\pageref{p.accel}).

\item En mode avancé, un événement permet de créer des événements accéléromètre en indiquant des valeurs pour les accéléromètres avant/arrière ou gauche/droite (page~\pageref{p.accel}).

\end{itemize}

\textbf{Modifications de l'interface utilisateur}

\begin{itemize}

\item Des actions multiples associées au même événement (page~\pageref{p.multiple}).

\item Les blocs et les paires événement-actions peuvent être copiées (page~\pageref{p.copy-pairs}).

\item Les captures d'écran des programmes VPL peuvent être exportés en plusieurs formats (page~\pageref{p.proximity-sensitivity}).

\item Les boutons annuler/rétablir ont été ajoutés (page~\pageref{p.undo}).

\item Le bouton Lancer clignote en vert lorsque le programme a été modifié (page~\pageref{p.blink}).

\item Il n'est désormais plus possible de modifier la palette de couleurs de VPL.

\end{itemize}
