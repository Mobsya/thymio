\chap{Plusieurs Thymio}\label{ch.two}

Vous pouvez contôler deux ou plus de Thymio simultanément.

\textbf{Instructions}

Placez deux robots T1 et T2, un en face de l'autre. T1 poursuit T2;
lorsque T1 détecte qu'il est proche de T2, il s'arrête.
Si T2 détecte que T1 est proche de lui, T2 recule jusqu'à ce qu'il ne voit plus T1.

\textbf{Conseils}

\begin{itemize}

\item Les programmes de T1 et T2 ont deux paires événements-action: l'événement de la première
est la détection d'un objet par le capteur central horizontal,
l'événement de la seconde est l'absence de détection d'un objet.
Mais les programmes de T1 et T2 diffèrent par les actions associées aux événements.

\item Connectez les deux Thymios T1 et T2 à l'ordinateur et allumez-les.
Lancez deux fois VPL. Dans la fenêtre qui permet de choisir le robot qu'utilisera VPL
(Figure~\ref{fig.connect}), vous devriez voir à la fois T1 et T2;
sélectionnez T1 dans la première instance de VPL et T2 dans la seconde.
Ouvrez et lancez le programme \p{chase} pour T1 et le programme \p{retreat} pour T2.

\end{itemize}

\textbf{Expériences}

\begin{itemize}

\item Que se passe-t-il si vous échangez les programmes: lancez \p{retreat} pour T1 et \p{chase} pour T2? Expliquez.

\item En mode avancé, testez différents paramètres des seuils des capteurs.

\end{itemize}

{\raggedleft \hfill Programme: \bu{chase.aesl}, \bu{retreat.aesl}}

\bigskip

\informationbox{Communiquer entre les robots}{Plusieurs robots Thymio peuvent s'envoyer des messages.
Cette fonctionnalité est disponible dans le langage AESL et l'environnement Studio.}
