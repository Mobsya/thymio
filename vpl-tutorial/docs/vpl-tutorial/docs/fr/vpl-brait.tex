\part{Projets}

\chap{Les créatures de Braitenberg}
\label{ch.brait}

\sect{Que sont les créatures de Braitenberg?}

\href{http://fr.wikipedia.org/wiki/Valentino_Braitenberg}{Valentino Braitenberg}
était un spécialiste en neurosciences qui créa des véhicules virtuels
qui présentaient des comportements étonnamment complexes.\footnote{V. Braitenberg.
\textit{Vehicles: Experiments in Synthetic Psychology} (MIT Press, 1984).}
Ses véhicules ont été largement adoptés dans la robotique éducative.
Des chercheurs du MIT Media Lab ont repris ces modèles pour construire ces véhicules en vrai.
Ils les ont appelés les \emph{Créatures de Braitenberg}.\footnote{David W. Hogg, Fred Martin,
Mitchel Resnick. \textit{Braitenberg Creatures}. MIT Media Laboratory, E\&L Memo 13, 1991.
\href{http://cosmo.nyu.edu/hogg/lego/braitenberg_vehicles.pdf}{http://cosmo.nyu.edu/hogg/lego/braitenberg\_vehicles.pdf} [en anglais].}
Ces véhicules étaient construits à partir de \emph{briques programmables} qui étaient les
précurseurs des kits robotiques LEGO Mindstorms.

Ce document donne une implémentation de la majorité des créatures de Braitenberg décrites dans
rapport le rapport du MIT, adaptées pour le robot Thymio avec la VPL.
Les créatures développées au MIT utilisaient des lumières et des capteurs tactiles
alors que Thymio utilise principalement les capteurs infrarouges de proximité.
Nous avons gardé les noms des créatures
qu'on leur avait donné dans le rapport du MIT pour pouvoir établir une équivalence même s'ils ne correspondent pas à l'implémentation pour le Thymio.
L'ordre aussi des descriptions des créatures a été conservé
bien que celui-ci ne corresponde pas à la difficulté de son implémentation dans VPL.

Dans les descriptions, on utilise, à part mention contraire,
l'expression <<\,détecte un objet\,>> pour signifier
que le robot détecte un objet par le capteur avant central.
Le plus simple pour générer cet événement est de placer sa main
devant le capteur central
de sorte qu'elle soit détectée par le capteur.

Le code source VPL est disponible dans l'archive.
Les fichiers portent les noms en anglais des créatures qu'ils représentent
plus l'extension \texttt{\small aesl}.
Les noms des fichiers apparaîtront entre parenthèses, à côté du nom du comportement en français.
Pour certaines créatures, des comportements supplémentaires sont proposés en exercice.
Vous trouverez leur implémentation aussi dans l'archive.

\sect{Les créatures}

\begin{description}

\item[Timide (timid.aesl)] Tant que le robot ne détecte aucun objet, il avance. 
Dès qu'un objet est détecté, il s'arrête.

\item[Indécis (indecisive.aesl)] Le robot avance tant qu'il ne détecte aucun objet.
Lorsqu'un objet est détecté, il recule.
Lorsqu'il se trouve à une certaine distance, le robot \emph{oscille}
entre marche avant et marche arrière.

\item[Paranoïaque (paranoid.aesl)] Lorsque le robot détecte un objet, il avance. S'il ne détecte
aucun objet, il tourne à gauche.

\textbf{Exercice (paranoid1.aesl)} 
Lorsque capteur central du robot détecte un objet, le robot avance.
Lorsque c'est le capteur droit (mais pas le capteur central) qui détecte un objet,
le robot tourne à droite.
Lorsque c'est le capteur gauche (mais pas le capteur central) qui détecte un objet,
le robot tourne à gauche.

\textbf{Exercice (paranoid2.aesl - mode avancé)} 
Tâche similaire à \textbf{Paranoïaque}, mais cette fois changer à chaque seconde la direction dans laquelle le robot tourne. \textbf{Indice}: Utilisez des états pour retenir la direction actuelle
et un minuteur pour changer d'état.

\item[Entêté (dogged.aesl)] Lorsqu'un objet est détecté à l'avant, le robot recule et
lorsqu'un objet est détecté à l'arrière, le robot avance.

\textbf{Exercice (dogged1.aesl)}
Reprenez \textbf{Entêté}, mais arrêtez le robot lorsqu'aucun objet n'est détecté.

%\textbf{Exercice (dogged2.aesl - mode avancé)}
%Le robot avance tant qu'il ne rencontre pas d'objet, puis recule.
%Il recule alors jusqu'à ce qu'il détecte un nouvel objet, puis avance de nouveau.
%\textbf{Indice} Utilisez les états pour retenir la direction actuelle.
%L'état change lorsqu'un objet est détecté et c'est l'état qui contrôle les moteurs.

\item[Désécurisé (insecure.aesl)]
Si le capteur gauche ne détecte pas d'objet, allumez le moteur droit du robot et éteignez le moteur gauche.
Si un objet est détecté par le capteur gauche, allumez le moteur gauche et éteignez le moteur droit.
Le robot devrait alors suivre un mur à sa gauche.
\textbf{Indice}: Voir la référence \cref{a.blocks} sur les virages avec Thymio.

\item[Déterminé (driven.aesl)]
Lorsque un objet est détecté par le capteur gauche, il allume le moteur droit et éteint le moteur gauche.
Et lorsque un objet est détecté par le capteur droit, il allume le moteur gauche et éteint le moteur droit.
Le robot devrait alors s'approcher de l'objet en zigzaguant.

\item[Insistant (persistent.aesl)]
Le robot avance jusqu'à ce qu'il détecte un objet.
Il recule alors pendant une seconde et ensuite avance de nouveau.

\item[Attiré et repoussé (attractive-repulsive.aesl)]
Lorsqu'un objet approche le robot, il s'enfuit jusqu'à ce qu'il ne le voit plus.

\item[Constant (consistent.aesl - mode avancé)]
Chaque fois qu'on donne une tappe au robot, celui-ci passe à l'état suivant:
d'abord il avance, ensuite il tourne à gauche, puis à droite, puis recule, puis recommemce.

%\item[Inhumain (inhumane.aesl)]
%Le robot continue à avancer jusqu'à ce qu'il rencontre un obstacle.

\item[Paniqué (frantic.aesl - mode avancé)]
La lumière du haut clignote en rouge.
\textbf{Indice}: Vous pouvez utiliser le bloc événement capteur avec tous les carrés des capteurs en mode gris comme décrit dans \cref{a.blocks}.

\textbf{Exercice (frantic1.aesl - mode avancé)}
Implémentez le clignotement à l'aide du bloc événement boutons au lieu du bloc événement capteurs.
Y a-t-il une différence de comportement du robot? Si oui, quelle en est la cause?

\item[Observateur (observant.aesl - mode avancé)]
La lumière du haut est verte quand le capteur droit détecte un objet.
La lumière du haut devient rouge lorsque le capteur gauche détecte un objet.
Une fois la lumière du haut allumée, le robot attend trois secondes avant de s'éteindre;
pendant ce temps, la lumière ne change pas.

\end{description}
