\chap{Aimer pour un temps (Mode avancé)}\label{ch.time}

Dans le \cref{ch.pet}, nous avons programmé un robot de compagnie qui nous aimait ou nous fuyait.
Considérons un comportement plus avancé : un compagnon timide qui n'arrive pas à se décider s'il nous aime ou pas.
Initiallement, le robot tournera en direction de notre main tendue, puis il s'en éloignera en tournant dans l'autre sens.
Après un moment, il changera d'avis et reviendra en direction de notre main.

{\raggedleft \hfill Programme: \bu{shy.aesl}}

Lorsque le bouton droite est touché, le robot tourne à droite:
\blkc{start-turn}
Quand il détecte votre main, il tourne à gauche: \blkc{turn-away} 

Le comportement de se retourner «\,après un moment\,» peut être décomposé en deux paires événement-actions :
\begin{itemize}

\item \emph{Quand} le robot commence à se détourner $\rightarrow$ \emph{démarrer un minuteur} de deux secondes.

\item \emph{Quand} le temps du minuteur est écoulé $\rightarrow$ \emph{tourner} à droite.

\end{itemize}

Nous avons besoin d'une nouvelle \emph{action} pour la première partie du comportement et d'un nouvel \emph{événement} pour la seconde partie.

L'action qui démarre un \emph{minuteur}
est semblable à un réveil \blksm{action-timer}.
Quand on demande au réveil de sonner à un moment précis,
on lui donne une heure absolue, par exemple 7 heures.
Mais quand j'utilise mon smartphone comme réveil,
il me donne aussi l'heure relative: <<\,alarme dans 11 heures et 23 minutes.\,>>
Le minuteur fonctionne ainsi.
Vous indiquez au bloc action minuteur un certain nombre de secondes; quand le nombre de secondes données seront écoulées,
le minuteur générera un événement minuteur.

On peut régler le minuteur jusqu'à quatre secondes.
Sur le bloc, chaque seconde est représentée par un quart de la montre.
Cliquez où vous souhaitez sur la montre;
une petite animation vous indiquera le temps que représente le compte à rebours;
ensuite, la partie correspondante de la montre sera colorée en bleu foncé.
%Ce minuteur est, en fait, un compte à rebours qui peut aller jusqu'à 4 secondes.
%Pour régler la durée du compte à rebours, vous pouvez cliquer n'importe où dans le cadran de l'alarme.
%Une petite animation vous montre combien de temps le compte à rebours va durer.

\informationbox{Le mode avancé}
{Les minuteurs sont disponibles en mode avancé.
Cliquez sur \blkmed{advanced} pour entrer en mode avancé.\\
L'icône deviendra alors \blkmed{basic}. En cliquant dessus, retournerez en mode débutant.}

La paire événement-actions pour cette première partie du comportement est : \blkc{turn-clock}

Lorsque l'événement détectant votre main se produit, il y aura donc deux actions : tourner le robot à gauche et lancer un compte à rebours de deux secondes.

La seconde partie du comportement nécessite un événement se produisant lorsque le compte à rebours arrive à zéro, c'est le bloc événement \emph{temps écoulé} \blksm{event-timer} qui montre un réveil sonnant.

La paire événement-actions pour faire tourner à nouveau le robot vers la droite lorsque le minuteur est écoulé est donc : 
\blkc{turn-back}

\bigskip

\exercisebox{\thechapter.1}{
Écrivez un programme qui fasse avancer le robot à vitesse maximale pour trois seconde lorsque que le bouton avant est touché ; puis qui fasse reculer le robot.
Ajoutez une paire événement-actions qui arrête le mouvement en touchant le bouton central.
}

\trickbox[Information]{En robotique, ce genre de compte à rebours s'appelle des \textit{timers}.
Ils sont extrêmement utiles dans de nombreuses situations et vous vous en apercevrez très rapidement en créant vos propres comportements pour Thymio.}
