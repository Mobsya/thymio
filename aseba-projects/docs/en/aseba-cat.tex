\part{Projects}

\chap{The Cat Catches the Mouse}\label{ch.cat}

\sect{Specification}

A cat turns left and right searching for a mouse. When it senses a
mouse, it turns in the direction of the mouse and pounces on it quickly.
When the cat collides with the mouse, it stops and announces its
success.

\begin{center}
\begin{picture}(255,80)
%\put(0,0){\framebox(255,80){}}
%  Cat turns
\put(0,0){
\put(0,0){\framebox(30,30){}}
\put(30,15){\oval(30,30)[r]}
\put(35,30){\vector(-1,1){15}}
}  
\put(10,70){
\put(10,0){\framebox(20,10){}}
\put(5,5){\circle{10}}
\put(30,5){\line(1,0){15}}
}
%  Cat faces mouse
\put(100,0){
\put(0,0){\framebox(30,30){}}
\put(15,30){\oval(30,30)[t]}
\put(15,45){\vector(0,1){15}}
}
\put(110,70){
\put(10,0){\framebox(20,10){}}
\put(5,5){\circle{10}}
\put(30,5){\line(1,0){15}}
}
%  Cat catches mouse
\put(200,25){
\put(0,0){\framebox(30,30){}}
\put(15,30){\oval(30,30)[t]}
%\put(15,45){\vector(0,1){15}}
}
\put(210,70){
\put(10,0){\framebox(20,10){}}
\put(5,5){\circle{10}}
\put(30,5){\line(1,0){15}}
}
\end{picture}
\end{center}

\sect{System design}

The Thymio robot will be the cat which searches for a toy
mouse. When the center button is touched, the robot turns left and then
right repeatedly searching for the mouse. When the mouse is detected by
one of the front horizontal proximity sensors, the robot turns until the
mouse is detected by the \emph{center} proximity sensor. Then it quickly moves
forward in that direction and stops when the mouse is very close. The
robot announces the cat's success by playing a victory tune: the first
four notes of Beethoven's Fifth Symphony. The robot displays its current
state in the top LEDs.

\sect{State machine}

The program transitions sequentially through four states:

\begin{center}
\unitlength=1.2pt
\begin{picture}(300,35)
%\put(0,0){\framebox(300,35){}}
\put(30,10){\oval(60,20)}
\put(110,10){\oval(60,20)}
\put(190,10){\oval(60,20)}
\put(270,10){\oval(60,20)}
\put(0,0){ \makebox(60,20){\bu{0=off}}}
\put(80,0){\makebox(60,20){\bu{1=search}}}
\put(160,0){\makebox(60,20){\bu{2=turn to}}}
\put(240,0){\makebox(60,20){\bu{3=pounce}}}
\put( 60,10){\vector(1,0){20}}
\put(140,10){\vector(1,0){20}}
\put(220,10){\vector(1,0){20}}
\put(270,20){\line(0,1){15}}
\put(30,35){\line(1,0){240}}
\put(30,35){\vector(0,-1){15}}
\end{picture}
\end{center}

\begin{itemize}

\item The transition from \bu{off} to \bu{search} occurs when
the center button is touched.

\item The transition from \bu{search} to \bu{turn to} occurs
when the mouse is detected.

\item The transition from \bu{turn to} to \bu{pounce} occurs
when the center sensor detects the mouse.

\item The transition from \bu{pounce} to \bu{off} occurs
when the center sensor detects that the mouse is very close.

\item Additionally, in all the states except \bu{off}, touching the
center button causes a transition to state \bu{off}. (These transitions
are not shown in the above diagram.)
\end{itemize}
The state \bu{search} has substates for turning left and right, as does
the playing of the sound.

\sect{Constants}

\begin{itemize}

\item \p{SCAN}: The power of the motors during the left and right search.
One wheel receives positive (forward) power of this magnitude, while the
other wheel receives negative (backwards) power of this magnitude.

\item \p{TIMER}: The time between changes of direction during the search.
\item \p{TURN}: The low power of the motors when turning to face the mouse.
\item \p{POUNCE}: The high power of motors when pouncing on the mouse.
\item \p{DETECTION}: The threshold at which a proximity sensor detects the mouse.
\item \p{COLLISION}: The threshold at which a proximity sensor detects a collision.
\end{itemize}

\sect{Variables}

\begin{itemize}
\item \p{state}: The current value of the state machine (0 .. 3). 
\item \p{motor\_state}: The direction (+1 = right, $-$1 = left) in which the cat is turning.
\item \p{sound\_state}: The state of playing the victory tune (0 .. 4).
\item \p{i}: An index variable for the loop that checks if the mouse is detected.

\end{itemize}

\sect{Event handlers and subroutines}
\begin{itemize}

\item Event center button is touched: The event occurs when the button
is \emph{released}. The subroutine \p{on\_off} is called.

\item Subroutine \p{on\_off}: The robot is turned on by setting the
state to 1 (\bu{search}) if it is 0 and turned off by setting the state
to 0 (\bu{off}) if it is not. The subroutine \p{initialize} is
called.

\item Subroutine \p{initialize} sets the initial values of the variables
(\p{state}, \p{sound\_state}, \p{motor\_state}), the motors, the LEDs and
the timer (\p{timer[0]}).

\item Event \bu{timer0} expires: If the robot is in state 1
(\bu{search}), subroutine \p{search} is called.

\item Subroutine \p{search} changes the direction of the motor from left
to right or from right to left. The five front sensors are
sampled (in a loop) to determine if the mouse is detected; if so, the
state is changed to 2 (\bu{turn to}).

\item Event proximity sensor sampled (10 times per second): If the
current state is 2 (\bu{turn to}), the subroutine \p{turn\_to} is
called, while if the current state is 3 (\bu{pounce}), the subroutine
\p{pounce} is called.

\item Subroutine \p{turn\_to}: If the mouse is detected by the central
sensor, the state is changed to 3 (\bu{pounce}) and the robot is driven
forward rapidly. Otherwise, if the mouse is detected by either of the
left sensors, the robot is slowly turned to the left. Otherwise, the
mouse is detected by either of the right sensors, the robot is slowly
turned to the right.

\item Subroutine \p{pounce}: If the central sensor detects that the
mouse is very close, the motors are stopped and the state is set to 0
(\bu{off}). The first note of the victory tune is played.

\item Event sound finishes: This event occurs when the note that is
currently being played has finished. The next note in the sequence is
played and the variable \p{sound\_state} is incremented, until all four
notes have been played.

\item LEDs: The LEDs on top of the robot are lit with distinct colors in
the following subroutines: \p{initialize}, \p{search}, \p{turn\_to},
\p{pounce}, and in the event handler \p{sound.finished}. The colors
depend on the direction of the robot and on which sensors detect the
mouse.

\end{itemize}

\sect{Programming notes}

{\raggedleft \hfill Program file \bu{cat.aesl}}

Start each program with a title, author and copyright notice:
\begin{verbatim}
  # Thymio program: cat hunts mouse
  # Copyright 2013 by Moti Ben-Ari
  # CreativeCommons BY-SA 3.0
\end{verbatim}

I suggest using the
\href{http://creativecommons.org/licenses/by-sa/3.0/}{CreativeCommons
BY-SA license}, which allows anyone to copy and modify your program,
provided that your name remains in the program and provided that they
agree to let other people share their work.

Next, write comments explaining the constants and declarations of the
variables with comments:
\begin{verbatim}
  # Constants
  # SCAN      - power of motors during search
  # TURN      - power of motors during turn towards mouse
  # POUNCE    - power of motors during pounce on mouse
  # TIMER     - time between change of directions
  # DETECTION - threshold for the detection of the mouse
  # COLLISION - threshold for the detection of a collision with the mouse

  # Variables
  var state         # 0 = off, 1 = search, 2 = turn to, 3 = pounce
  var motor_state   # +1 = right, -1 = left
  var sound_state   # 0-4 for playing Beethoven's Fifth
  var i             # loop variable
\end{verbatim}

The values of the comments are set in the upper right corner of Aseba
Studio, but comments cannot be added directly so you should document
them within the program.

The initialization should be comprehensive since you don't know what the
current state of the robot is when you run the program. For example, I
turned off the temperature LED so that it won't confuse me when I
observe the display on the robot:

\begin{verbatim}
  call leds.temperature(0,0)
\end{verbatim}

I use a lot of subroutines to keep the size of program components small
and easy to read. In particular, I do very little in the event handlers
themselves and put most of the processing in a subroutine:
\begin{verbatim}
  onevent prox
    if state == 2 then
      callsub turn_to
    elseif  state == 3 then
      callsub pounce
    end
\end{verbatim}

The order of alternatives in an \p{if}-statement is important and you
may have to experiment to get the best results. The center sensor is
checked before the side ones:

\begin{verbatim}
  sub turn_to
    if     (prox.horizontal[2] > DETECTION) then
               # pounce
    elseif (prox.horizontal[0] > DETECTION) or
           (prox.horizontal[1] > DETECTION) then
               # turn right
    elseif (prox.horizontal[3] > DETECTION) or
           (prox.horizontal[4] > DETECTION) then
               # turn left
    end
\end{verbatim}
This means that if the center sensor \emph{and} another one detect the
mouse in the same sample, the robot starts to pounce. Alternatively,
if the center sensor were checked as the last alternative in the
\p{if}-statement, the robot would pounce \emph{only} when the center
sensor alone detects the mouse. This is more precise but in the real
world it might not occur.


\sect{Experiments}

\begin{enumerate}

\item Experiment with ``mice'' of different sizes,
colors and materials (metal, plastic).

\item Experiment with different combinations of sensors. In my program,
both right and both left sensors are checked to see if they detect the
mouse. What happens if only the rightmost and leftmost sensors are
checked? What happens if the robot pounces on the mouse when it is
detected by the center sensor \emph{and} one of the side sensors?

\item You may want to define different thresholds for different sensors.

\item Experiment with the speeds \p{SCAN}, \p{TURN} and \p{POUNCE} to
see how they affect the ability of the robot to detect and pounce on the
mouse.

\end{enumerate}
