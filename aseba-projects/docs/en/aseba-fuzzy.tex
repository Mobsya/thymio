\chap{Fuzzy Logic Control}\label{ch.fuzzy}

The control algorithms in the previous chapter used mathematical
computations. An alternate approach is to use
\href{http://en.wikipedia.org/wiki/Fuzzy_logic}{\emph{Fuzzy logic}}, which was originally proposed by
\href{http://en.wikipedia.org/wiki/Lotfi_A._Zadeh}{Lotfi A. Zadeh} in
1965. In fuzzy logic, the control algorithms use \emph{rules}:

\begin{itemize}
\item If the \emph{temperature is low} and the \emph{humidity is high},
set the \emph{heater to high}.
\item If the \emph{temperature is low} and the \emph{humidity is low},
set the \emph{heater to medium}.
\end{itemize}

The logic is ``fuzzy'' because the rules are expressed in terms of
\emph{linguistic variables} like \emph{temperature} which do not have
precise mathematical specifications, but only imprecise linguistic
specifications like \emph{high}.

In this chapter we implement a fuzzy logic controller for
the task from the previous chapter of approaching an object.

\centeredbox{This project is complex. It assumes that you have studied
an introduction to fuzzy logic, for example, from
\href{http://blog.peltarion.com/2006/10/25/fuzzy-math-part-1-the-theory/}{this
online tutorial} or from Section~2.2 of Kevin M. Passino and Stephen
Yurkovich. \textit{Fuzzy Control}, Addison-Wesley, 1998. }

\sect{Overview}

A fuzzy logic controller consists of three phases that are run
 sequentially:

\begin{description}

\item[Fuzzify] The values of the sensors are converted into values of
linguistic variables called \emph{premises}. Each premise specifies the
\emph{certainty} with which we believe that the variable is true. There
will be three variables: \p{far}, \p{closing}, \p{near}.

\item[Apply rules] A set of \emph{rules} expresses the control
algorithm. Given a set of premises, a \emph{consequent} is inferred.
There will be five variables for the consequents: \p{very fast},
\p{fast}, \p{cruise}, \p{slow}, \p{stop}.

\item[Defuzzify] The consequents are combined in order to produce an
\emph{crisp} output, which for our project is the power applied to the
motors.

\end{description}

\centeredbox{Certainties of the premises and consequents are given in
the range 0.0 (not at all certain) through 1.0 (completely certain).
Since Aseba does not support floating-point numbers, they will be scaled
to the range 0 through 100.}


\sect{Fuzzify}

When approaching an object, the value read by the center horizontal
sensor increases from 0 to a large number: over 4000, depending on your
robot and the reflecting properties of the object.\footnote{It is a good
idea to use reflecting tape as noted on page~\pageref{p.reflect}.} The
precise value returned by the sensor needs to be fuzzified---converted
to the value of a linguistic variable. Figure~\ref{fig.fuzz} shows three
graphs for converting the sensor values into certainties of the
variables for \p{far}, \p{closing} and \p{near}. The $x$-axis is the
value returned by the sensor and the $y$-axis is the certainty.

\begin{figure}
\begin{center}
\begin{picture}(265,120)
%\put(0,0){\framebox(265,120){}}
\put(15,15){
\drawline(0,0)(250,0)
\drawline(0,0)(0,100)
\drawline(0,100)(50,100)
\dashline{4}(50,100)(50,0)
\drawline(50,100)(100,0)
\drawline(75,0)(125,100)
%\dashline{4}(125,100)(125,0)
\drawline(125,100)(175,0)
\drawline(150,0)(200,100)
\dashline{4}(200,100)(200,0)
\drawline(200,100)(250,100)
\put(50,0){\circle*{4}}
\put(75,0){\circle*{4}}
\put(100,0){\circle*{4}}
%\put(125,0){\circle*{4}}
\put(150,0){\circle*{4}}
\put(175,0){\circle*{4}}
\put(200,0){\circle*{4}}
}
\put(15,0){
\put(45,0){\makebox(10,10){\textsf{1}}}
\put(70,0){\makebox(10,10){\textsf{2}}}
\put(95,0){\makebox(10,10){\textsf{3}}}
\put(145,0){\makebox(10,10){\textsf{4}}}
\put(170,0){\makebox(10,10){\textsf{5}}}
\put(195,0){\makebox(10,10){\textsf{6}}}
\put(0,0){\makebox(10,10){\textsf{(0)}}}
\put(220,0){\makebox(10,10){\textsf{(4000)}}}

\put(20,75){\makebox(10,10){\textsf{far}}}
\put(120,75){\makebox(10,10){\textsf{closing}}}
\put(220,75){\makebox(10,10){\textsf{near}}}
}
\put(0,10){\makebox(10,10){\textsf{0}}}
\put(0,110){\makebox(10,10){\textsf{100}}}
\end{picture}
\caption{Fuzzify the center horizontal sensor}\label{fig.fuzz}
\end{center}
\end{figure}

The labeled points on the $x$-axis refer to thresholds: \p{(1) FAR\_LOW,
(2) CLOSING\_LOW, (3) FAR\_HIGH, (4) NEAR\_LOW, (5) CLOSING\_HIGH, (6)
NEAR\_HIGH}. If the value of the sensor is below \p{(1) FAR\_LOW}, then
we are completely certain that the object is far away. If the value is
between \p{(2) CLOSING\_LOW} and \p{(3) FAR\_HIGH} then we are somewhat
certain that the object is far away but also somewhat certain that it is
closing. The fuzziness results from the overlapping ranges: when the
value is between point (2) and point(3), we can't say with complete
certainty if the object is far away or closing.

\newpage

\sect{Apply rules}

The three premises---the certainties of \p{far}, \p{closing} and
\p{near}---are used to compute five consequents using the following
rules:

\begin{enumerate}
\item If \p{far} then \p{very fast}
\item If \p{far} and \p{closing} then \p{fast}
\item If \p{closing} then \p{cruise}
\item If \p{closing} and \p{near} then \p{slow}
\item If \p{near} then \p{stop}
\end{enumerate}

The certainties of the consequents resulting from rules 1, 3, 5 are the
same as the certainties of the corresponding premises. When there are
two premises, as in rules 2 and 4, the certainties of the consequents
are the minimum of the certainties of the premises. Since \emph{both} of
the premises must apply, we can't be more certain of the consequent than
we are of the smaller of the premises.


\sect{Defuzzify}

The next step is to combine the consequents, taking into account their
certainties. Figure~\ref{fig.defuzz} shows the output motor powers for
each of the five consequents. For example, if we were completely certain
that the output is \p{cruise}, the center graph in the figure shows that
the motor power would be set to 200, but if we were less certain, the
motor power would be less or more.

\begin{figure}
\begin{center}
\begin{picture}(315,120)
%\put(0,0){\framebox(315,120){}}
\put(15,15){
\drawline(0,0)(300,0)
\drawline(0,0)(0,100)
\multiput(0,0)(50,0){5}{
\drawline(0,0)(50,100)
\drawline(50,100)(100,0)
\put(50,0){\circle*{4}}
\dashline{4}(50,0)(50,100)
}
}
\put(10,0){
\put(50,0){\makebox(10,10){\textsf{0}}}
\put(100,0){\makebox(10,10){\textsf{100}}}
\put(150,0){\makebox(10,10){\textsf{200}}}
\put(200,0){\makebox(10,10){\textsf{300}}}
\put(250,0){\makebox(10,10){\textsf{400}}}
}
\put(0,10){\makebox(10,10){\textsf{0}}}
\put(0,110){\makebox(10,10){\textsf{100}}}
\end{picture}
\caption{Defuzzify to obtain the crisp motor setting}\label{fig.defuzz}
\end{center}
\end{figure}

For our rules more than one consequent can have positive values. These
are combined using the areas of the trapezoids defined by the
certainties. Figure~\ref{fig.trap} shows the consequent \p{cruise} with
a certainty of 40 and the consequent \p{fast} with a certainty of 20.

\begin{figure}
\begin{center}
\begin{picture}(315,120)
%\put(0,0){\framebox(315,120){}}
\put(15,15){
\drawline(0,0)(300,0)
\drawline(0,0)(0,100)
\multiput(100,0)(50,0){2}{
\drawline(0,0)(50,100)
\drawline(50,100)(100,0)
}
\drawline(120,40)(180,40)
\drawline(160,20)(240,20)
\put(100,35){\makebox(10,10){\textsf{40}}}
\put(250,15){\makebox(10,10){\textsf{20}}}
}
\put(10,0){
\put(50,0){\makebox(10,10){\textsf{0}}}
\put(100,0){\makebox(10,10){\textsf{100}}}
\put(150,0){\makebox(10,10){\textsf{200}}}
\put(200,0){\makebox(10,10){\textsf{300}}}
\put(250,0){\makebox(10,10){\textsf{400}}}
}
\put(0,10){\makebox(10,10){\textsf{0}}}
\put(0,110){\makebox(10,10){\textsf{100}}}
\end{picture}
\caption{Areas defined by the certainties of the consequents}\label{fig.trap}
\end{center}
\end{figure}

Let $w$ and $h$ be the width and height of the triangle. Then the area
of the trapezoid bounded by the certainty $h'$ is given by the
formula:\footnote{The derivation of the formula is given in an appendix
to this chapter.}

\begin{displaymath}
wh'-wh'(\frac{h'}{2h}).
\end{displaymath}

For $w=h=100$ and $h'=20,40$, the areas are:

\begin{displaymath}
\begin{array}{l}
100\cdot 20 - 100\cdot 20\cdot\frac{20}{200} = 1800\\
100\cdot 40 - 100\cdot 40\cdot\frac{40}{200} = 3200.\\
\end{array}
\end{displaymath}

To obtain a crisp value, the \emph{center of gravity} is computed. This
is the sum of the areas weighted by the center of each triangle divided
by the sum of the areas. For the example---and removing the scaling
factor of 100---this gives:

\begin{displaymath}
\frac{18\cdot 3 + 32\cdot 2}{18+32}=39.
\end{displaymath}
This is the crisp value can now be scaled to the range 0
through 500 of the motors.

\sect{State machine}

There are two states: 0 for \p{off} and 1 for \p{on}.

\sect{Constants}

Define the six thresholds as shown in Figure~\ref{fig.fuzz}.

\sect{Variables}

The main variables are a three-element array \p{premises} and a
five-element array \p{consequents}. Another five-element array contains
the \p{centers} of the output membership functions. The output is stored
in the variable \p{crisp}.

\sect{Event handlers and subroutines}

\begin{itemize}

\item The state is turned on and off using the front and center button event handlers.

\item When a proximity event occurs, three subroutines \p{fuzzify},
\p{apply\_rules}, \p{defuzzify} are called in sequence, and then the
value of \p{crisp} is assigned to the motor targets.

\item Subroutine \p{fuzzify} checks the value of the center horizontal
sensor against the threshold values. The certainties defined by the
functions in Figure~\ref{fig.fuzz} are stored in the array \p{premises}.

\item Subroutine \p{apply\_rules} checks for non-zero values in
\p{premises} and then assigns certainties to the array \p{consequents}
as described above.

\item Subroutine \p{defuzzify} computes the areas and center of gravity
to obtain a crisp value.

\end{itemize}

\sect{Programming notes}

\begin{itemize}

\item You can't really see how the algorithm works by observing the
robot. Instead, observe the variables in the table at the left of the
window. Even better, assign the values of the premises and consequents
to the circle LEDs after calling \p{apply\_rules}. Since the certainties
are scaled to the range 0 through 100, by dividing by three, the
certainties can be seen from the intensity of the LEDs:

\begin{verbatim}
call leds.circle(premises[0]/3, premises[1]/3, premises[2]/3,
                 consequents[0]/3, consequents[1]/3, consequents[2]/3,
                 consequents[3]/3, consequents[4]/3)
\end{verbatim}

\item Scale the values appropriately since floating-point
arithmetic is not supported.

\item The native function \p{math.multdiv} is very useful for carrying
out the computations.

\end{itemize}

{\raggedleft \hfill Program file \bu{fuzzy.aesl}}

\newpage

\sect{Appendix: Computing the area from the certainty}

The following diagram shows a triangle of width $w$ and height $h$ cut off at height $h'$:

\begin{center}
\begin{picture}(170,110)
%\put(0,0){\framebox(170,110){}}
\put(25,10){
\drawline(0,0)(100,0)
\drawline(0,0)(50,100)
\drawline(50,100)(100,0)
\drawline(15,30)(85,30)
\put(0,-10){\makebox(100,10){$w$}}
\put(0,30){\makebox(100,10){$w'$}}

\put(-10,0){\vector(0,1){100}}
\put(-25,45){\makebox(10,10)[r]{$h$}}
\put(110,0){\vector(0,1){30}}
\put(110,30){\vector(0,1){70}}
\put(115,15){\makebox(10,10)[l]{$h'$}}
\put(115,65){\makebox(10,10)[l]{$h-h'$}}
}
\end{picture}
\end{center}

The area of the trapezoid is the difference
between the areas of the two triangles:
\begin{displaymath}
a = wh/2 - w'(h-h')/2.
\end{displaymath}
By similar triangles:
\begin{displaymath}
\frac{h}{h-h'} = \frac{w}{w'},
\end{displaymath}
so:
\begin{displaymath}
w' = \frac{w(h-h')}{h}
\end{displaymath}

Substituting:
\begin{eqnarray*}
a &=& \frac{wh}{2} - \frac{w(h-h')(h-h')}{2h}\\
&=&\frac{w(h^2-(h-h')^2)}{2h}\\
&=&\frac{w(h^2-h^2+2hh'-h'^2)}{2h}\\
&=&\frac{w(2hh'-h'^2)}{2h}\\
%&=&w(h'-\frac{h'^2}{2h})\\
&=&wh'-wh'(\frac{h'}{2h}).\\
\end{eqnarray*}