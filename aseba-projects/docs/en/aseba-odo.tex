\chap{Odometry}\label{ch.odo}

Suppose that you are in a car and your GPS navigation system
instructs you: ``In 700 meters turn right.'' You no longer need the
system to choose the correct turn. Simply check the car's
\emph{odometer} which measures how far you have traveled and when it
approaches 700 meters beyond its initial reading, look for a road
on the right. \emph{Odometry}---the measurement of distance---is a
fundamental method used by robots for navigation.
However, even with a relatively accurate odometer, you will
want to look carefully for an intersection, because you don't want to
turn even two or three meters too early or too late.

Odometers measure time and speed,
and then compute distance traveled by multiplying them. In
this chapter we demonstrate odometry with the Thymio robot.

Feel free to make the programs more exciting by including lights
and sounds.

\sect{Specification}

The Thymio robot travels straight at a constant speed for a fixed
period of time. Compute how far it has traveled and compare the
computed distance with the actual distance.

\sect{Design}

The motor power is set by assigning values to
\p{motor.X.target} (where \p{X} is \p{left} or \p{right}). The internal
computer of the robot tries to maintain this power level, but it is
possible to measure the actual power level by reading the values of
\p{motor.left.speed} and \p{motor.right.speed}. In theory, once we set
the motor power, it will remain constant, but in practice it changes
frequently, so we will sample the actual power level at frequent
intervals. The distance traveled during each interval will be computed
and these distances will be added to obtain the total distance
traveled by the robot.

The values returned by \p{motor.X.speed} are in the range 0--500 and
have to be converted into speeds measured in centimeters per second.
Implement the project in Chapter~\ref{ch.fast}
to compute the relationship between
power levels and speed of your Thymio.

From the table on page~\pageref{tab.speed},
I determined that my Thymio travels over the floor
at about $\frac{\mathit{Power}}{100}\times 3.2$ cm/sec. Multiply this by
the duration of the sampling interval in seconds to obtain the distance
that the robot has traveled during that interval. For example, if the
value in \p{motor.X.speed} is 350 and the interval is 0.5 seconds,
the distance traveled is $\frac{350}{100}\times 3.2 \times 0.5 = 5.6$ cm.

\sect{Constants}

\begin{itemize}
\item \p{MOTOR}: The motor target speed
\item \p{DELTA}: The time between samples
\item \p{COUNT}: The number of samples
\item \p{SPEED}: The speed of robot per 100 power setting
\end{itemize}

\sect{Programming}

{\raggedleft \hfill Program file \bu{odo1.aesl}}

\begin{itemize}

\item The actual motor speeds should be measured starting from the \emph{second}
occurrence of the timer event so that the robot has time to achieve its target speed.

\item Sample the speed of either the left or the right motor on the assumption
that they are roughly the same. Alternatively, sample both speeds
and take their average.

\item The distance traveled during an interval is
$\Delta d=\frac{p}{100} \times s \times \Delta t$,
where $p$ is the sampled speed (power),
$s$ is the power-to-speed conversion factor and
$\Delta t$ is the time length of the interval.

Since real numbers are not supported,
$s=3.2$ cm/sec and $t=500 \; \textrm{msec} =0.5 \: \textrm{sec}$ are scaled
by multiplying by ten: $s'=32$ \textbf{mm/sec} and $t'=5$ \textbf{sec}.
The distance is then divided twice by ten.
To preserve the precision of $\Delta d$,
the division should be done on the total distance
obtained by summing each $\Delta d$.

\item Use the native function for 32-bit computation (see Section~\ref{s.on-off}):
\begin{verbatim}
    call math.muldiv(deltaD, p, s'*t', 100)
\end{verbatim}

\end{itemize}

\sect{Running the program}

\begin{itemize}
\item Run the program several times, measure the distances traveled by the Thymio,
and compare them with the computed distances. How accurate is the odometry
computation? How consistent are the measured and computed distances?

\item Does the robot travel straight? If it turns to one side, you might want to
add or subtract a value from the constant \p{MOTOR} when assigning it as
the target speed to one of the motors.

\item Place an object at a known distance in front of the Thymio.
Modify the program so that when the odometry computation determines
that the robot is near the object, it detects the object using
the horizontal sensors and then moves slowly until the robot touches the object.

\end{itemize}

\sect{Experiments}

{\raggedleft \hfill Program file \bu{odo2.aesl}}

The motors are set to run at a constant speed,
so we would probably get reasonably accurate measurements of the distance
by taking a single sample.
Modify the program so that the robot \emph{randomly} changes the target speeds
of the motors in each interval after the speed is sampled.

The native function \p{math.rand} returns a random
number in the range $-32768 .. 32767$. Before using this value as a target
speed you will have to perform a computation to ensure that the value is in
the range of the motor power values $0 .. 500$.
