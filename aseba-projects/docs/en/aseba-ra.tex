\chap{Distributed Mutual Exclusion}\label{ch.ra}

Place two Thymios facing each other with a reflective object about
halfway between them:

\begin{center}
\begin{picture}(230,45)
%\put(0,0){\framebox(230,45){}}
\put(0,30){\makebox(30,15){\textsf T1}}
\put(0,0){\framebox(30,30){}}
\put(30,15){\oval(30,30)[r]}
\put(200,30){\makebox(30,15){\textsf T2}}
\put(200,0){\framebox(30,30){}}
\put(200,15){\oval(30,30)[l]}
\put(115,15){\circle{30}}
\end{picture}
\end{center}

What will happen if we run the following program in both robots?

\begin{verbatim}
onevent button.forward
  motor.left.target = 100
  motor.right.target = 100

onevent prox
  if prox.horizontal[2] > 4000 then
    motor.left.target = 0
    motor.right.target = 0
  end
\end{verbatim}

{\raggedleft \hfill Program file \bu{no-mutex.aesl}}

If both forward buttons are touched at about the same time, the two
robots will simultaneously move towards the object; each robot will
stop when it is close to the object.

\sect{Distributed algorithms}

Suppose that we want the robots to approach the object one-by-one. This
is called the \emph{critical section problem} or the \emph{mutual
exclusion problem}. It is a fundamental problem the area of computing
called \emph{concurrent and distributed programming}. Obviously, we can
achieve this goal by touching the button of one robot, waiting until it
has reached the object and then touching the button of the other robot.
The difficulty with this solution is that it assumes a central computing
device that makes decisions for both robots. This is unrealistic in
practice: consider the Internet where millions of computing devices
synchronize and communicate with no central control. The basic
requirement for distributed algorithms is that the identical algorithm
run on all participating computing devices and that the access to
services is fair, that is, some devices will not receive preferential
treatment.

\sect{The Ricart-Agrawala algorithm}

We now describe a \emph{distributed} algorithm for mutual
exclusion.\footnote{The algorithm is based on the Ricart-Agrawala
algorithm, which achieves mutual exclusion for any number of nodes
repeatedly trying to enter a critical section. The algorithm it takes
into account several difficulties that we ignore here for simplicity,
for example, the possibility that two robots choose the same random
number. A full explanation can be found in Chapter~10 of M. Ben-Ari.
\textit{Principles of Concurrent and Distributed Programming (Second
Edition)}, Addison-Wesley, 2006.} How do people synchronize access to
ticket counters, food service stations, and so on? They take slips of
paper with numbers and are served in numerical order, thus implementing
a queue. The queuing is facilitated by an electronic display, but a
queue can be implemented simply by having people compare numbers with
each other.

The two robots choose \emph{random} numbers:

\begin{center}
\begin{picture}(230,45)
%\put(0,0){\framebox(230,45){}}
\put(0,30){\makebox(30,15){\textsf T1}}
\put(0,0){\framebox(30,30){47}}
\put(30,15){\oval(30,30)[r]}
\put(200,30){\makebox(30,15){\textsf T2}}
\put(200,0){\framebox(30,30){23}}
\put(200,15){\oval(30,30)[l]}
\put(115,15){\circle{30}}
\end{picture}
\end{center}

Each robots sends its number in a message to the other robot:

\begin{itemize}
\item \textsf{T1} receives 23 from \textsf{T2} and compares it with
its own number 47. Since \textsf{T2} has the lower number, it should be
served before \textsf{T1}. Therefore, \textsf{T1} sends a \emph{reply}
to \textsf{T2} to indicate that it can approach the object.

\item \textsf{T2} receives the reply from \textsf{T1} and approaches
the object. \emph{At the same time}, \textsf{T2} has received the
message from \textsf{T1} with its number. Since 47 is greater than 23,
\textsf{T2} \emph{does not} send a reply to \textsf{T1}; instead,
it records that the sending of the reply is \emph{deferred}.
Since \textsf{T1} does not receive a reply, it does not approach the
object.

\item When \textsf{T2} is close to the object it stops. It also recalls
that it has deferred the reply to \textsf{T1} and sends it now.

\item \textsf{T1} receives the reply and approaches the object.  
\end{itemize}

It is important to note that \emph{both} robots run the same program and
that the algorithm is fair, in the sense that if run repeatedly, the
random numbers will give priority to each robot about half the time.

\sect{State machine}

Touching the forward button indicates that the robot wants to approach
the object. It sends a request and starts moving when the reply is
received:

\begin{center}
\unitlength=1.2pt
\begin{picture}(300,35)
%\put(0,0){\framebox(300,35){}}
\put(30,10){\oval(60,20)}
\put(110,10){\oval(60,20)}
\put(190,10){\oval(60,20)}
\put(270,10){\oval(60,20)}
\put(0,0){ \makebox(60,20){\bu{0=initial}}}
\put(80,0){\makebox(60,20){\shortstack{\bu{1=want to}\\\bu{approach}}}}
\put(160,0){\makebox(60,20){\shortstack{\bu{2=request}\\\bu{sent}}}}
\put(240,0){\makebox(60,20){\shortstack{\bu{3=reply}\\\bu{received}}}}
\put( 60,10){\vector(1,0){20}}
\put(140,10){\vector(1,0){20}}
\put(220,10){\vector(1,0){20}}
\put(270,20){\line(0,1){15}}
\put(30,35){\line(1,0){240}}
\put(30,35){\vector(0,-1){15}}
\end{picture}
\end{center}

\sect{Variables}
\begin{itemize}
\item \p{number}: The number chosen by this robot (0--1022).
The value 1023 is used for the reply message.
\item \p{state}: The current state of the state machine.
\item \p{deferred}: Set to 1 if the reply must be deferred.
\end{itemize}

\sect{Programming notes}

The event handler for the center button will stop the motors, enable
IRC, and set initial values of the variables, including \p{prox.comm.tx}
and \p{prox.comm.rx}. The event handler for the forward button will set
\p{state} to 1 to initiate the algorithm.

Most of the processing is done in the event handler for IRC
communication. In state 1, a random number is chosen and sent to the
other Thymio:

\begin{verbatim}
onevent prox.comm
  if state == 1 then
      call math.rand(number)
      number = abs(number) &amp; 1022
      prox.comm.tx = number
      state = 2
      return
  end
\end{verbatim}

In state 2, the \p{prox.comm.rx} is checked to see if it is the reply
message (1023); if so, the motors are started:
\begin{verbatim}
  if state == 2 then
    if prox.comm.rx == 1023 then
      motor.left.target = 100
      motor.right.target = 100
      state = 3
      return
    end
  end
\end{verbatim}

If the robot is not competing to approach the object (it is in state 0)
or if a lower number is received from the other robot, the reply is
sent; otherwise it is deferred:
 
\begin{verbatim}
  if state == 0 or prox.comm.rx < number then
    prox.comm.tx = 1023
  else
    deferred = 1
  end
\end{verbatim}

When the robot is close to the object the motors can be turned off. The
variable \p{deferred} is checked to see if a reply must be sent to the
other robot:

\begin{verbatim}
onevent prox
  if prox.horizontal[2] > 4000 then
    motor.left.target = 0
    motor.right.target = 0
    if deferred == 1 then
      prox.comm.tx = 1023
    else
      prox.comm.tx = 0	
    end
  end
\end{verbatim}

{\raggedleft \hfill Program file \bu{ra.aesl}}

\sect{Testing the program}

Place the robots and an object as shown in the above
diagram.\footnote{The robots have to be relatively close so that IRC
works and slightly off-center so that the object doesn't obstruct IRC.}
Load and run the program into both robots. Touch the center buttons to
initialize. Touch the forward button on one of the robots; only that
robot should approach the object. Move the robot back to its starting
position and touch the center button to reset. Now touch both forward
buttons at the same time. One robot should approach and stop, and then
the other robot should approach and stop.
