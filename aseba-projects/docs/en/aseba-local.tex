\chap{Localization}\label{ch.local}

\newcommand*{\doors}{
\put(-5,-10){\framebox(330,50){}}
\put(0,  0){\colorbox{dark-gray}{\makebox(30,30){}}}
\put(40, 0){\colorbox{dark-gray}{\makebox(30,30){}}}
\put(80, 0){\colorbox{light-gray}{\makebox(30,30){}}}
\put(120, 0){\colorbox{light-gray}{\makebox(30,30){}}}
\put(160,0){\colorbox{dark-gray}{\makebox(30,30){}}}
\put(200,0){\colorbox{dark-gray}{\makebox(30,30){}}}
\put(240,0){\colorbox{dark-gray}{\makebox(30,30){}}}
\put(280, 0){\colorbox{light-gray}{\makebox(30,30){}}}

\multiputlist(20,-25)(40,0){\textsf 0,1,2,3,4,5,6,7}
}

\newcommand*{\pbar}[1]{
\drawline(0,0)(0,#1)
\drawline(0,#1)(30,#1)
\drawline(30,#1)(30,0)
\put(0,#1){\makebox(30,15){.#1}}
}

\newcommand*{\paxes}{
\drawline(0,0)(330,0)
\drawline(0,0)(0,100)
\put(-15,45){\makebox(10,10){$p$}}
\put(-15,0){\makebox(10,10){$0.0$}}
\put(-15,90){\makebox(10,10){$1.0$}}
\put(0,-30){\makebox(330,10){$x$}}
\multiputlist(25,-10)(40,0){\textsf 0,1,2,3,4,5,6,7}
}

\sect{How can the robot know where it is?}

Consider a robot that is navigating within a known environment for which
it has a \emph{map}. The following map shows a wall with five doors
(dark gray blocks) and three areas where there is no door (light gray
blocks):

\begin{center}
\begin{picture}(370,65)
%\put(0,0){\framebox(370,65){}}
\put(40,25){\doors}
\end{picture}
\end{center}

The task of the robot is to enter a specific door, say the one at
position 4. But how can the robot know where it is? In
Chapters~\ref{ch.odo},~\ref{ch.odo-advanced}, you learned about
odometry, which enables a robot to determine its current position given
a know starting position. For example, if the robot is at the left end
of the wall:

\begin{center}
\begin{picture}(370,65)
%\put(0,0){\framebox(370,65){}}
\put(40,25){
\doors
\put(-40,-10){\framebox(30,30){}}
\put(-10,5){\oval(30,30)[r]}
}
\end{picture}
\end{center}

it knows that it has to travel four times the width of each door, while
if the robot is at the following position:

\begin{center}
\begin{picture}(370,65)
%\put(0,0){\framebox(370,65){}}
\put(40,25){
\doors
\put(80,-10){\framebox(30,30){}}
\put(110,5){\oval(30,30)[r]}
}
\end{picture}
\end{center}

the required door is the next one to the right. Unfortunately, odometry
is subject to error in the measurement of the robot's speed and
direction, so it is quite likely that as time goes by the robot will get
lost. The purpose of \emph{localization} algorithms is to determine the
robot's position within a known environment. In this chapter, we
implement a simple one-dimensional version of the \emph{Markov
algorithm}.

\centeredbox{This chapter follows the explanations of the algorithms
 given in the first week of the online course \emph{Artificial
 Intelligence for Robotics} by Sebastian Thrun, which can be found at
\href{https://www.udacity.com/course/artificial-intelligence-for-robotics--cs373}%
{https://www.udacity.com/course/artificial-intelligence-for-robotics--cs373}.}

\sect{Sensing increases certainty}

The robot assigns a probability to the eight positions where it might be
located. Initially, it has no idea where it is, so each position will be
assigned the probability $1.0/8=.125$:\footnote{The labels on the bars
in the graphs in this chapter are rounded.}

\begin{center}
\begin{picture}(350,130)
\put(15,30){
\paxes
\multiput(10,0)(40,0){8}{\pbar{13}}
}
\end{picture}
\end{center}

Suppose now that the robot uses its sensors and detects a dark gray
area. Now its uncertainty is reduced, because it knows that it is in
front of one of the five doors. The probability distribution shows
$.2$ for each of the doors and $.0$ for each of the walls:

\begin{center}
\begin{picture}(350,130)
\put(15,30){
\paxes
\multiput(10,0)(40,0){2}{\pbar{20}}
\multiput(170,0)(40,0){3}{\pbar{20}}
\put(90,0){\pbar{0}}
\put(130,0){\pbar{0}}
\put(290,0){\pbar{0}}
}
\end{picture}
\end{center}

Next our mobile robot moves forwards (to the right) and again senses a
dark gray area. There are now only three possibilities: it was at
position 0 and moved to 1, it was at 4 and moved to 5, or it was at 5
and moved to 6. If the robot's initial position was 6, after moving
right it would no longer detect a dark gray area, so it could not have
been there. The probability is now $.33$ for each of these three
positions:

\begin{center}
\begin{picture}(350,130)
\put(15,30){
\paxes
\put(10,0){\pbar{0}}
\put(50,0){\pbar{33}}
\multiput(90,0)(40,0){3}{\pbar{0}}
\multiput(210,0)(40,0){2}{\pbar{33}}
\put(290,0){\pbar{0}}
}
\end{picture}
\end{center}

After the next step of the robot, if it again detects a door, it is
without doubt at position 6, while if it does not detect a door, it is
either at position 2 or position 7.

\sect{Uncertainty in sensing}

The difference between a dark gray door and a light gray wall is
not too great, and as the paint ages it might become more and more
difficult to distinguish between them. Furthermore, as you have
experienced, the robot's sensors do not give uniform responses and they
can change over time. Therefore, the robot cannot detect with
complete certainty whether is senses a door or a wall.

We will model this by assigning probabilities to the detection. If the
robot senses dark gray, we specify that the probability is $.9$ that it
has (correctly) detected a door and $.1$ that it has (mistakenly)
detected a wall. Conversely, if it senses light gray, the probabilities
are $.1$ for a door and $.9$ for a wall.

After sensing dark gray at a position where there is a door, we only
know with probability $.125\times .9 = .1125$ that a door has been
correctly detected; however, there is still a $.125\times .1= .0125$
possibility that it has mistakenly detected a door. After normalizing
(see sidebar), the probability distribution is:

\begin{center}
\begin{picture}(350,130)
\put(15,30){
\paxes
\multiput(10,0)(40,0){2}{\pbar{19}}
\multiput(90,0)(40,0){2}{\pbar{02}}
\multiput(170,0)(40,0){3}{\pbar{19}}
\put(290,0){\pbar{02}}
}
\end{picture}
\end{center}

\centeredbox{The set of probabilities of the possible outcomes of an
event must add up to $1$ since one of the outcomes must occur. The
probabilities that we computed add up to much less than $1$:
\[.1125 + .1125 + .0125 + .0125 + .1125 + .1125 + .1125 + .0125 = .575.\]
We must \emph{normalize} the probabilities by dividing by the sum
$.575$ so that the sum will be $1$. We have $.1125/.575\approx .19$ and 
$.0125/.575\approx .02$, so:
\[.19 + .19 + .02 + .02 + .19 + .19 + .19 + .02 \approx 1.\]}

The robot now moves to the right and we will assume that the
probability distribution is cyclic, that is, the value associated
with position 7 becomes the value associated with position 0:

\begin{center}
\begin{picture}(350,130)
\put(15,30){
\paxes
\put(10,0){\pbar{02}}
\multiput(50,0)(40,0){2}{\pbar{19}}
\multiput(130,0)(40,0){2}{\pbar{02}}
\multiput(210,0)(40,0){3}{\pbar{19}}
}
\end{picture}
\end{center}

If the robot again senses dark gray, the probability of being at
positions 1, 5 or 6 should increase. Computing the probabilities
by multiplying and normalizing gives:

\begin{center}
\begin{picture}(350,130)
\put(15,30){
\paxes
\multiput(10,0)(80,0){3}{\pbar{3}}
\put(50,0){\pbar{29}}
\put(130,0){\pbar{0}}
\multiput(210,0)(40,0){2}{\pbar{29}}
\put(290,0){\pbar{3}}
}
\end{picture}
\end{center}

If the robot moves right again and senses a third dark gray area,
the distribution becomes:

\begin{center}
\begin{picture}(350,130)
\put(15,30){
\paxes
\multiput(10,0)(40,0){3}{\pbar{7}}
\put(130,0){\pbar{1}}
\put(170,0){\pbar{0}}
\put(210,0){\pbar{7}}
\put(250,0){\pbar{64}}
\put(290,0){\pbar{7}}
}
\end{picture}
\end{center}

Not surprisingly, the robot is almost certainly at position 6.

\sect{Specification}

In order the careful examine the computations, our initial
implementation will not move the robot. Instead, we will place the robot
over black and white areas on the table and touch a button: first, to
cause the robot to read the ground sensors and to compute the new
probability distribution, and then to move the distribution (cyclically)
to the right. The values of the variables can be examined in the
\textbf{Variables} table (see page~\ref{p.variables}).

The values of the probability distributions will also be displayed in
the circle LEDs. The robot will be localized---will have computed its
position with a high degree of certainty---when exactly one LED is
bright and the others are dim.

\sect{System design}

There are two subroutines: \p{sense} which computes the changes in the
probabilities as a result of reading the sensors, and \p{move} which
virtually moves the robot one position to the right.

\sect{State machine}

There are two states that remember whether \p{sense} or \p{move} is to
be performed next.

\sect{Constants}
\begin{itemize}
\item LED: A scale factor for displaying probabilities in the LEDs.

\item THRESHOLD: The value below which black is detected.

\item P\_HIT: The probability of detecting black when on a black area or
detecting white when on a white area.
              
\item P\_MISS: The probability of detecting black when on a white area
or detecting white when on a black area.

\end{itemize}


\sect{Variables}

\begin{itemize}

\item \p{world}: An 8-element array with the \emph{map} of the
environment. It is initialized by \p{[1, 1, 0, 0, 1, 1, 1, 0]},
corresponding to the diagram at the beginning of the chapter, where 1
means that a door exists at that position and 0 means that a wall exists
at that position.

\item \p{beliefs}: An 8-element array with the current probability
distribution of what the robot believes is its position. The array is
initialized by the uniform distribution with 125 in each element.

\item \p{state}: State variable 0 = sense, 1 = move.

\item Auxiliary variables: \p{i}, \p{hit}, \p{temp}, \p{sum}.

\end{itemize}


\sect{Event handlers and subroutines}

\begin{itemize}
\item Event \p{button.center}: Initialize \p{beliefs} to the uniform
distribution and set \p{state} to 0.

\item Event \p{button.forward}: Call subroutine \p{sense} or \p{move}
depending on the state and update the value of \p{state}.

\item Subroutine \p{display\_beliefs}: Display the values of the
elements of \p{beliefs} in the circle LEDs after scaling approximately
to the range 0 through 32.

\item Subroutine \p{sense}: Perform the following operations:

\begin{itemize}
\item If both ground sensors are below the \p{THRESHOLD}, detect a door,
otherwise detect a wall.

\item For each position in \p{world}, if the detection matches the
element (door or wall), multiply the corresponding position in
\p{beliefs} by \p{P\_HIT}; otherwise, multiply by \p{P\_MISS}.

\item Normalize the elements in \p{beliefs} by dividing each one by
the sum of the elements.

\end{itemize}

\item Subroutine \p{move}: Move elements of \p{beliefs} one position to
the right cyclically.

\end{itemize}

\sect{Programming notes}

\begin{itemize}

\item The probabilities in \p{beliefs} are scaled to the range 0 through 1000.

\item The factors \p{P\_HIT} and \p{P\_MISS} are scaled to the range 0
through 10.

\item Use the native function \p{math.multdiv} to carry out the
computations.

\end{itemize}

{\raggedleft \hfill Program file \bu{local-simple.aesl}}

\sect{Experiments}

\begin{enumerate}

\item We showed how three measurements are sufficient to localize the
robot if it starts at position 4. Try other positions and see if the
algorithm works.

\item Experiment with the values of \p{P\_HIT} and \p{P\_MISS}. How low
can you specify \p{P\_HIT} and have the algorithm still work?

\end{enumerate}


\sect{Uncertain motion}

As well as uncertainty in the sensors, robots are subject to uncertainty
in their motion. We can ask the robot to move one ``position'' to the
right, but it might move two positions, or it might move very little and
remain in its current position. Let us modify the program for
localization to take this uncertainty into account.

Let $b$ be the belief array. The subroutine \p{sense} computed $b'$, the
new values of the elements, using the formula:\footnote{The computations
must take account of the cyclic movement.}
\begin{displaymath}
b'_i = b_i \times p,
\end{displaymath}
where $p$ is \p{P\_HIT} or \p{P\_MISS}, depending on whether the
sensor value matches the map or not. The subroutine \p{move}
simply moved the values to the right $b'_i = b_{i-1}$. With uncertainty,
\p{move} must perform the following computation:
\begin{displaymath}
b'_i = (b_{i-2}\times p_2) + (b_{i-1}\times p_1) + (b_{i} \times p_0),
\end{displaymath}
where $p_n$ is the probability that the robot has moved $n$ positions:

\begin{center}
\begin{picture}(130,90)
%\put(0,0){\framebox(130,90){}}
\put(0,60){
\put(0,0){\framebox(30,30){$b_{i-2}$}}
\put(15,0){\vector(3,-1){87}}
\put(50,0){\framebox(30,30){$b_{i-1}$}}
\put(65,0){\vector(3,-2){44}}
\put(100,0){\framebox(30,30){$b_{i}$}}
\put(115,0){\vector(0,-1){30}}
}
\put(120,40){\makebox(10,10){$p_0$}}
\put(95,40){\makebox(10,10){$p_1$}}
\put(25,40){\makebox(10,10){$p_2$}}
\put(0,0){\framebox(30,30){}}
\put(50,0){\framebox(30,30){}}
\put(100,0){\framebox(30,30){$b'_{i}$}}

\end{picture}
\end{center}


It is highly likely that the robot will move correctly, so reasonable
values are $p_1=0.8$ and $p_0=p_2=0.1$. (Don't forget to normalize
after performing the computation.) With these values for uncertainty, the
beliefs after sensing three black areas are:
\begin{center}
\begin{picture}(350,130)
\put(15,30){
\paxes
\put(10,0){\pbar{11}}
\put(50,0){\pbar{17}}
\put(90,0){\pbar{04}}
\put(130,0){\pbar{01}}
\put(170,0){\pbar{02}}
\put(210,0){\pbar{13}}
\put(250,0){\pbar{46}}
\put(290,0){\pbar{05}}
}
\end{picture}
\end{center}

The robot is likely at position 6, but it is less certain than before
(.46 instead of .64).

\sect{Programming notes}

Declare an array \p{save} of the same length as \p{beliefs} and copy to
\p{save} (using the native function \p{math.copy}) the values of
\p{beliefs} before starting the computation. The original values are
thus remain available.

{\raggedleft \hfill Program file \bu{local-uncertain-motion.aesl}}


\sect{Localization of a moving robot}

Let us now attempt localization of a moving robot. Place three strips of
tape at regular intervals on a table to simulate the three doors at
positions 4, 5 and 6. Set the motors to a constant speed and set a timer
to the amount of time it takes the robot to travel from one strip to the
next. When the timer event occurs, call the subroutines \p{sense} and
\p{move}. Start the robot before the first strip and see if it locates
itself at position 6 after sensing the third strip.

{\raggedleft \hfill Program file \bu{local-move.aesl}}


\sect{Appendix: The mathematics of localization}

\textbf{Bayes rule}

The problem of localization is to determine the probability $p(x_i)$
that the robot is at position $x_i$ for all $i$. In particular, given
the reading $z$ of the sensor, what is the probability that we are at
position $x_i$? This is expressed as a \emph{conditional probability}
$p(x_i\mid z)$. The data we have available are the \emph{current}
probability $p(x_i)$ that we are at $x_i$, and $p(z\mid x_i)$, the
conditional probability that the sensor reads $z$ if we are in fact at
$x_i$. In the example, initially, $p(x_i)=.125$ for all positions $x_i$,
and, if we are at a door, the probability that the sensor detects this
correctly is $.9$. We can now \emph{multiply} these two probabilities to
obtain (the unnormalized) probability $\bar{p}(x_i\mid z)$, in the
example, $.125\times .9=.1125$.

To normalize the probability, we have to divide it by the sum of the
results obtained by multiplying at each position:

\begin{displaymath}
p(z) = \sum_i p(z\mid x_i)\cdot p(x_i).
\end{displaymath}

In the example, the sum is $.575$ and $.1125/.575=.19$.

The formula:
\begin{displaymath}
p(x\mid z) = \frac{p(z\mid x)\cdot p(x)}{p(z)}
\end{displaymath}
is called \emph{Bayes rule}.


\textbf{Total probability}

To compute the change in probability when the robot moves, we used
the formula:
\begin{displaymath}
b'_i = (b_{i-2}\cdot p_2) + (b_{i-1}\cdot p_1) + (b_{i} \cdot p_0).
\end{displaymath}
Expressed in terms of probability, this is:
\begin{displaymath}
p'(x_i) = p(x_{i-2})\cdot p(x_i\mid x_{i-2}) \; + \;
p(x_{i-1})\cdot p(x_i\mid x_{i-1}) \; + \; p(x_{i})\cdot p(x_i\mid x_{i}),
\end{displaymath}
where $p(x_i\mid x_j)$ is the probability that the current position is
$x_i$ given that the previous position was $x_j$. The general case is:
\begin{displaymath}
p'(x_i) = \sum_j \;p(x_j)\cdot p(x_i\mid x_j),
\end{displaymath}
which is called the \emph{total probability}.
